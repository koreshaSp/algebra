\section{Лекция 07.03}
\subsection{Ранг линейного отображения}
\begin{definition}
    Будем называть линейной системой координат изоморфизм между каким-либо векторным пространством
    $V$ и пространством столбцов $K^n$. (Далее изоморфизм будет обозначаться как $V \simeq K^n$).
\end{definition}
\begin{definition}
    Определим образ линейного отображения $L\colon U\mapsto V$ как $\im L = \{Lx\in V \mid x \in U\}$. 
\end{definition}
\begin{remark}
    Пусть $L: U\mapsto V$~--- линейное отображение. 
    Тогда $\im L \leq V$.
\end{remark}
\begin{definition}
    Определим ядро линейного отображения $L\colon U\mapsto V$ как $\ker L = \{x\in U \mid Lx = 0\}$. 
\end{definition}
\begin{remark}
    Пусть $L: U\mapsto V$~--- линейное отображение. 
    Тогда $\ker L \leq U$.
\end{remark}
\begin{definition}
    Рангом линейного отображения назовём размерность образа линейного отображения.
    $\rk L = \dim \im L$
\end{definition}
\begin{theorem}[О подходящем выборе базиса]
    Пусть $L: U\mapsto V$, $k = \rk L, n = \dim U$. 
    Тогда $\exists e_1,\dots,e_n$~--- базис $U$ такой, что
    $Le_1,\dots, Le_k$~--- базис $\im L$, а $e_{k+1},\dots,e_{n}$~--- базис $\ker L$
\end{theorem}
\begin{proof}
    Обозначим за $m=n - \dim \ker L$.
    Далее построим $e_{m+1},\dots, e_n$~--- базис $\ker L$.
    Тут имеется ввиду, что мы строим базис с нуля, но индексируем немного странным
    образом в связи с тем, чтобы далее было удобно показать, что на самом деле $m = k$.

    Теперь пользуясь \hyperref[thm:О дополнении до базиса]{теоремой о дополнении до базиса}
    дополним наш базис
    при помощи любой порождающей системы $U$ до базиса $U$. Теперь для доказательства
    исходной теоремы нам необходимо доказать, что только что добавленные $e_1,\dots, e_m$~--- базис $\im L$.
    Будем делать это по определению, а именно докажем порождаемость и линейную независимость.

    \begin{itemize}
        \item Порождаемость.\\
            Заметим, что $Le_1,\dots,Le_n$~--- порождает $L$(просто из-за того, что $e_1,\dots,e_n$~--- базис $U$).
            Но все $Le_{m+1}, \dots, Le_n$ равны нулю, а это значит, что эта система равносильна
            $Le_1,\dots,Le_m$, откуда следует, что $Le_1,\dots, Le_m$~--- порождающая $L$.
        \item Линейная независимость.\\
            Для доказательства этой части возьмём произвольную линейную комбинацию равную 0 и докажем её 
            тривиальность(то, что $\lambda_i = 0, \forall i$).

            $\sum\limits_{1}^{m}{\lambda_i Le_i} = 0$. Пользуясь линейностью $L$ равенство можно записать
            в виде: $L\sum\limits_{1}^{m}{\lambda_ie_i} = 0$. Заметим, что сумма просто по определению
            лежит в $\ker L$. Это, в свою очередь значит, что эта же сумма представима как линейная
            комбинация элементов из базиса $\ker L$, то есть:  
            $L\sum\limits_{1}^{m}{\lambda_ie_i} = L\sum\limits_{m+1}^{n}{\lambda_ie_i} = 0$. 
            Теперь можно перенести оба выражения в левую часть, чтобы чётко видеть, что мы получили
            просто линейную комбинацию базиса $U$, равную нулю: 
            $L\left(\sum\limits_{1}^{m}{\lambda_ie_i} - \sum\limits_{m+1}^{n}{\lambda_ie_i}\right) = 0$. 
            Но так, как $e_1,\dots, e_n$~--- базис $U$, то такое возможно только когда
            $\lambda_i = 0, \forall i$, что и требовалось доказать.
    \end{itemize}
    Значит $Le_1,\dots,Le_m$~--- базис пространства $\im L$.
\end{proof}
\begin{follow}
    $L: U\mapsto V$~--- линейное отображение, причём $\dim U > \dim V$.
    Тогда выполнено:  $\dim \ker L > 0$.\\
\end{follow}
\begin{follow}
    $L: U\mapsto V$~--- линейное отображение, тогда $\dim \im L + \dim \ker L = \rk L + \dim \ker L = \dim U$
\end{follow}
\quad\\
\begin{statement}
    $L\text{~--- инъекция} \Leftrightarrow \dim \ker L = 0$.
\end{statement}
\begin{proof}\leavevmode\\
    $\ora$:\\
        Докажем от обратного, пусть $L$~--- инъективное, но $\dim \ker L > 0$.
        Тогда верно: $\exists u\in U\colon \lambda Lu = 0, \forall \lambda$.
        Значит можно просто взять $u, 2u$ и окажется, что  $Lu = L2u = 0$, 
        противоречие с инъективностью $L$.
    \\$\ola$:\\
        Пусть $L$ не инъективное и $\dim \ker L = 0$, тогда $\exists u_1, u_2\in U\colon Lu_1 = Lu_2\Rightarrow
        L(u_1 - u_2) = 0 \Rightarrow \dim \ker L > 0$, противоречие. 
\end{proof} 
\begin{follow}
    \label{принцип Дирехле}
    Пусть $\dim U = \dim V = n$, $L\colon U\mapsto V$~--- линейное.
    Тогда верна следующая цепочка утверждений:
    $$L\text{~--- инъективное} \Leftrightarrow \dim \ker L = 0 \Leftrightarrow \rk L = n \Leftrightarrow 
    \dim \im L = \dim V \Leftrightarrow L\text{~--- сюръективное}$$
    Получили нечто очень похожее на обобщённый принцип Дирехле для обыкновенной теории множеств.
\end{follow}

\subsection{Матрица системы уравнений}
\begin{motivation}
    Давайте лучше поймём какими наборами уравнений могут задаваться подпространства.
\end{motivation}
\begin{definition}
    Допустим $U\leq K^n$. Тогда будем говорить, что $U$ задаётся системой уравнений, если
    $\exists A \in M_{d\times n}(K)\colon \ker A = U$.
\end{definition}
\begin{remark}
    Сразу должен возникать вопрос: какие должны быть ограничения на $d$, чтобы задать в точности $U$?
\end{remark}
\begin{statement}
    Пусть $m = \dim U, U\leq K^n$ Тогда $U$ задаётся не менее $n - m$ уравнениями, при этом
    точно есть способ задать $U$ при помощи ровно $n - m$ уравнений.
\end{statement}
\begin{proof}
    Обозначим $d$~--- количество уравнений в системе. Докажем первую часть утверждения от обратного. 
    Пусть $A\in M_{d\times n}(K)\colon \ker Ax = U$, причём $d < n - m$.
    Зная, что $Ax\colon K^n\mapsto K^d$ уже можно получить какую-то оценку на $d$, 
    а именно $\dim \ker Ax \geq n - d$. Тогда можно перенести $n$ в левую часть и получить:
    $d \geq n - \ker Ax =  n - m$. Получили противоречие.

    Для того, чтобы доказать существование просто построим линейное отображение $L: K^n\mapsto K^{n-m}\colon
    \ker L = U$. Для этого удобнее всего воспользоваться 
    \hyperref[thm:Линейное отображение однозначно задаётся двумя базисами]{теоремой о единственности линейного
    отображения, которое задано парой базисов}.

    Построим базис пространства $K^n$ следующим образом:  $e_1,\dots,e_m$~--- базис $U$,
    $e_{m + 1},\dots, e_n$~--- его дополнение до $K^n$, такое существует по 
    \hyperref[thm:О дополнении до базиса]{теореме о дополнении до базиса}.
    Заметим, что $Le_1 = \dots = Le_m = 0$, так как иначе не выполняется условие: $\ker L = U$. 
    Тогда, оставшиеся $e_{m + 1},\dots, e_n$~--- должны быть базисом $\im L = K^{n - m}$.
    Самый простой способ добиться этого~--- отправить их в стандартный базис.
    
    Таким образом, мы определили наше линейное отображение на всех элементах базиса
    следующим образом $Le_1 = \dots = Le_m = 0$, $Le_{m + 1} = E_1, \dots, Le_n = E_{n - m}$,
    где $E_i$~--- $i$ый элемент стандартного базиса. А значит, по 
    \hyperref[thm:Линейное отображение однозначно задаётся двумя базисами]
    {теореме о том, что отображение однозначно задаётся двумя базисами}
    мы однозначно определили искомое линейное отображение $L$.
\end{proof}
\subsection{Про матрицы линейных отображений}
\begin{definition}
    Пусть $v\in V$,  $f_1,\dots, f_n$~--- базис $V$.
    Тогда координатами вектора $v$ в базисе $f$ называется следующее: 
    \[
        [v]_f =
        \begin{pmatrix}
            \lambda_1\\ \vdots\\ \lambda_n
        \end{pmatrix},
    \]
     где $\lambda_i$~--- коэффициенты в следующем разложении:
    $v = \sum\limits_{1}^{n}{\lambda_if_i}$.
\end{definition}
\begin{definition}
    Пусть $L: U\mapsto V$~--- линейное отображение, $e_1,\dots, e_n$~--- базис $U$, 
    $f_1,\dots, f_m$~--- базис $V$. $A = \Big([Le_1]_f, \dots, [Le_n]_f\Big)\in M_{m\times n}(K)$
    $A$~--- матрица линейного отображения $L$ в базисах $e, f$.
    Будем обозначать её следующим образом $[L]_f^e$
\end{definition}
\begin{remark}
    Хочется напомнить, что ранее мы поняли, что существует биекция между
    матрицами нужного размера и всеми линейными отображениями при условии, что мы
    знаем какие-либо базисы наших пространств.
\end{remark}
\begin{motivation}
    Будем развивать эту историю с матрицей отображения и поймём зачем она вообще нужна.
\end{motivation}
\begin{statement}
    $L: U\mapsto V, e_1,\dots, e_n\text{~--- базис }U, f_1,\dots, f_m\text{~--- базис }V$.
    $A \in M_{m\times n}(K)\colon \forall u\in U\colon A[u]_e = [Lu]_f$ равносильно тому,
    что $A$~--- матрица линейного отображения $[L]^e_f$.
\end{statement}
\begin{proof}\leavevmode\\
    $\ola$:\\
    Раскроем по определению: $[L]^e_f[u]_e=[Lu]_f$.
    Можно просто технически расписать по определению, но давайте проведём доказательство,
    которое даст лучшее понимание происходящего. Ключевой момент: $[u]_e, [Lu]_f$ в этой
    формуле лучше воспринимать просто как две биекции в $K^n$ и в $K^m$ соответственно.

    Тогда мы тут имеем систему следующих отображений:
    $K^n \simeq U \xrightarrow{L} V \simeq K^m$,
    $K^n\xrightarrow{[L]^e_f} K^m$.

    Зная, что, благодаря 
    \hyperref[thm:Линейное отображение однозначно задаётся двумя базисами]
    {теореме о том, что отображение однозначно задаётся двумя базисами}
    , можно задать линейное отображение, только на базисных векторах, посмотрим на то, чем
    будет являться $E_1\in K^n$ при переходе из $K^n$ в $K^m$ разными методами.
    \[
        E_1\in K^n \xrightarrow{([x]_e)^{-1}} e_1 \in U \xrightarrow{L} Le_1 \in V
        \xrightarrow{[x]_f} [Le_1]_f\in K^m
    .\]
    С другой стороны:
    \[
        E_1\in K^n\xrightarrow{[L]^e_fx} \Big([L]^e_f\Big)_{*, 1} = [Le_1]_f\in K^m
    .\] 
    Очевидно, что подобный трюк можно провернуть с любым $E_i$, а значит мы для всех
    элементов базиса $K^n$ верно $[L]^e_f[u]_e=[Lu]_f$, откуда следует, что это равенство
    верно и на всём пространстве $K^n$.\\
    $\ora$:
    Для доказательства в эту сторону теперь достаточно показать, что такая матрица $A$~---
    единственная. Пусть $\exists A,B\in M_{m\times n}(K)\colon A \not = B$, но при этом 
    $Ax = Bx \forall x\in U\simeq K^n$, сразу видно, что тут противоречие с $A\not=B$.
\end{proof}
\subsection{Связь умножения матриц и композиции линейных отображений}
\begin{statement}[О свойстве матриц линейных отображений]\leavevmode\\
    Пусть у нас есть следующая картина: $U\xrightarrow{L_1} V \xrightarrow{L_2} W$, $e, f, g$ 
    ~--- базисы в $U, V, W$ соответственно.
    \begin{enumerate}
        \item $[\lambda L]^e_f=\lambda[L]^e_f, \forall \lambda\in K$ 
        \item $[L_1 + L_2]^e_f = [L_1]^e_f + [L_2]^e_f$ 
        \item $[L_2\circ L_1]^e_g = [L_2]^f_g[L_1]^e_f$
    \end{enumerate}
\end{statement}
\begin{proof}
    Будет доказан только пункт 3, первые два остаются в качестве упражнения.
    Пусть $A = [L_1]^e_f, B = [L_2]^f_g$. Тогда распишем матрицу композиции.
    \[
        [L_2\circ L_1]^e_g[u]_e= [L_2(L_1u)]_g = B[L_1u]_f = B(A[u]_e) = (BA)[u]_e
    \]
\end{proof}
\begin{examples}
    \begin{enumerate}
        \item
            Если $K^n\mapsto^{L} K^m, L(x) = Ax$, пусть $E_i, F_i$~--- стандартные базисы $K^n, K^m$
            соответственно.
            Тогда $[L]^E_F = A$.
        \item
            Если $L: K[x]_{<n}\mapsto K[x]_{<n}, L(f(x)) = f(x + 1)$.
            
            Давайте посмотрим на то, во что переходит стандартный базис $K[x]$ :
            $x^i\xrightarrow{L}(x+1)^i = \sum\limits_{j=0}^{i}{\binom{i}{j}x^j}$.
            Имея эту формулу уже можно явно выписать матрицу отображения $L$,
            в ситуации, где за оба базиса выбраны стандартный базис $K[x]$.
            \[
            \begin{pmatrix}
                1&1&1&1&\dots&1\\
                0&1&2&3&\dots&n-1\\
                \vdots&\ddots&1&3&\dots&\binom{n-1}{2}\\
                \vdots&\ddots&\ddots&1&4&\dots&\\
                \vdots&\ddots&\ddots&\ddots&\ddots&\vdots\\
                0&\cdots&\cdots&\cdots&0&1\\
            \end{pmatrix}
        .\] 
        Нетрудно заметить, что мы получили треугольник Паскаля, но записанный в виде 
        верхнетреугольной матрицы.

        Теперь давайте попробуем выбрать другой базис для области значений и понаблюдаем,
        как это изменит матрицу отображения $L$. То есть работаем с базисами:
        $1, x,\dots, x^{n-1}$ для области определения и $1,(x + 1), \dots, (x+1)^n$ для
        области значений.  В данном случае достаточно заметить, что $i$-ый базисный
        вектор одного пространства переходит в $i$-ый базисный элемент другого, а это
        значит, что в этом случае матрица отображения будет просто единичной матрицой.
        \[
        \begin{pmatrix}
            1&0&\cdots&0\\
            0&1&\rotatebox{140}{\ldots}&\vdots\\
            \vdots&\rotatebox{140}{\ldots}&\rotatebox{140}{\ldots}&0\\
            0&\cdots&0&1\\
        \end{pmatrix}
        .\] 
    \end{enumerate}
\end{examples}
\begin{motivation}
    Теперь мы уже готовы понять, в чём интересность ранга отображения.
    А именно: выбор базисных векторов для построения матрицы линейного отображения
    никак не влияет на ранг отображения.
\end{motivation}
\subsection{Понятие обратимости}
\begin{motivation}
    На данный момент у нас есть 2 различных понятия обратимости.
    \begin{enumerate}
        \item Взгляд со стороны отображений, когда, говоря, что отображение обратимо,
            имеется ввиду, что существует обратное к нему отображение(что на
            самом деле эквивалентно тому, что отображение инъективное в случае,
            когда $L$~--- оператор).
        \item Взгляд со стороны алгебры, когда у нас есть какая-то алгебраическая
            операция $L$ над объектом и мы можем задать другую алгебраическую $L^{-1}$
            операцию такого же типа, для которой верно $(L^{-1}\circ L)x = x$.
    \end{enumerate}
    Цель этого раздела: связать эти два понятия.
\end{motivation}
\begin{definition}
    Матрица $A$ называется обратимой, если заданное ей линейное отображение $x \rightarrow Ax$, 
    является обратимым.
\end{definition}
\begin{definition}
    Матрица $A\in M_{m\times n}$ называется обратимой если существует матрица 
    $B\in M_{n\times m}\colon BA = E_n \text{ и } AB = E_m$.
\end{definition}
\begin{statement}\leavevmode
    \begin{enumerate}
        \item Два определения выше эквивалентны
        \item Все обратимые матрицы являются квадратными.
    \end{enumerate}
\end{statement}
\begin{proof}
    Пусть $e,f$~--- стандартные базисы $K^n, K^m$ соответственно.  

    Пусть матрица $A=[L]^e_f$~--- обратима по первому определению, это означает следующее:
    $\exists L^{-1}: K^m\mapsto K^n$, 
    причём $L^{-1}$~--- линейное по \hyperref[stm:Об обратном отображении линейного]
    {утверждению об обратном отображении линейного},
    доказанному ранее. А то, что $L^{-1}$~--- линейное обозначает, существует
    матрица $B = [L^{-1}]^f_e$.

    Зная, что $L^{-1}\circ L = id$, можно сделать вывод, что 
    $[L^{-1}\circ L]_e^e = E_m$.
    Тогда, по \hyperref[stm:О свойстве матриц линеных отображений]
    {утверждению о свойстве матриц линейных отображений}
    о свойстве матриц линейных отображений верно следующее равенство:
    $[L^{-1}]^f_e[L]^e_f = [L^{-1}\circ L]_e^e = E_n$. На самом деле 
    мы получили что $A$~--- обратимая матрица по второму определению, что
    нам и надо было доказать. Это отлично видно, если посмотреть, как задавались
    матрицы $A, B$. Получили $BA = E_n$. Аналогичное утверждение можно проделать с
    $L\circ L^{-1}$ и получить $AB = E_m$. 
    
    На данный момент формально доказано что если $A$~--- обратима по первому определению, то она
    обратима и по второму определению. Но на самом деле, можно убедиться, что все шаги доказательства
    верны в обе стороны, и можно утверждать, что утверждение в обратную сторону доказано автоматически,
    значит первую часть утверждения можно считать завершённой.

    Теперь мы готовы доказать первую часть.\\
    $$Ax\text{~--- обратимое линейное отображение} \Rightarrow
    Ax\text{~--- изоморфизм}\Rightarrow \abs{K^n = K^m} \Rightarrow {n = m}$$
\end{proof}
\begin{statement}[Базовый критерий обратимости]
    $A\in M_n(K)\text{~--- обратима} \Leftrightarrow \rk A = n \Leftrightarrow \dim \ker A = 0$
\end{statement}
\begin{proof}
    Очевидно следует из доказанного ранее \hyperref[принцип Дирехле]{аналога принципа Дирехле}.
\end{proof}
\begin{remark}
    Далее обратная матрица будет обозначаться как $A^{-1}$.
\end{remark}
\subsection{Координатизация}
\begin{definition}
    Матрицей преобразования координат из базиса $e$ в базис $f$ называется следующая матрица:
    $[id]^e_f$. Её главное свойство  $\forall u\in U, [id]^e_f[u]_e = [u]_f$.
\end{definition}
\begin{remark}
    Самый простой способ получить такую матрицу это написать
    следующее: $$\Big([e_1]_f,[e_2]_f,\dots,[e_n]_f\Big)$$
\end{remark}
\begin{motivation}
    Заметим, что это определение слегка не практично, так как часто базис $e$
    устроен проще, чем базис $f$(а зачастую $e$ и вовсе стандартный). 
    Но это определение весьма удобно для перевода из одного базиса в другой.
\end{motivation}
\begin{definition}
    Матрицей перехода из базиса $e$ в базис $f$ назовём следующую матрицу:
    $[id]^f_e$.
\end{definition}
\begin{properties}
    \item $[id]^f_g[id]^e_f = [id]^e_g$
    \item $[id]^e_e = E_n$ 
    \item $[id]^e_f=([id]^f_e)^{-1}$
\end{properties}
\begin{theorem}[О выборе базисов для получения элементарной матрицы линейного отображения]
    Пусть $L: U\mapsto V, n = \dim U, m = \dim V$.
    Тогда $\exists e, f$~--- базисы $U,V$ соответственно такие, что
    \[
    [L]^e_f = 
    \left(\begin{array}{c|c}
        E_r & 0\\
        \hline
        0 & 0
    \end{array}\right)\in M_{m\times n},\text{где } r = \rk L.
    \]
\end{theorem}
\begin{proof}
    Выбираем базис $e_1,\dots, e_n$, таким образом, что $Le_1,\dots,Le_r$~--- базис $\im L$,
    а $e_{r+1},\dots, e_n$~--- базис $\ker L$. То, что мы так можем сделать 
    \hyperref[thm:О подходящем выборе базиса]{доказывалось ранее}.
    Теперь выберем базис $f_1,\dots, f_m$ так:
    $f_1 = Le_1,\dots, f_r = Le_r$, а оставшиеся $f_{r+1},\dots, f_m$ определим так,
    чтобы дополнить первую часть до базиса $V$, это также 
    \hyperref[thm:О дополнении до базиса]{доказывалось ранее}.

    Нетрудно убедиться, что такой выбор базисов удовлетворяет условию теоремы.
\end{proof}
