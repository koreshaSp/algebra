\section{Лекция 1}

\subsection{Умножение матриц}
\quad
\begin{remark}
    Пусть есть $a_1 \cdot x_1 + a_2 \cdot x_2 + ... + a_n \cdot x_n = b$ и мы хотим записать это выражение \\ \\ компактно.
    Тогда удобно было бы, чтобы $(a_1, ..., a_n) \cdot 
    \begin{pmatrix}
        x_1\\
        \dots\\
        x_n\\
    \end{pmatrix} = a_1 \cdot x_1 + a_2 \cdot x_2 + ... + a_n \cdot x_n = b$.\\
\end{remark}

\begin{definition}
    \\
    $R$ - кольцо, $A \in M_{m \times n}(R), x \in M_{n \times 1}(R) = R^n \Longrightarrow
    \begin{pmatrix}
        a_{1*}\\
        \dots\\
        a_{m*}\\
    \end{pmatrix} \cdot x =   \begin{pmatrix}
        a_{1*} \cdot x\\
        \dots\\
        a_{m*} \cdot x\\
    \end{pmatrix} \in R ^ m = M_{m \times 1}$ 
\end{definition}

\quad
\begin{remark}
    $Ax = b$ --- удобная запись системы уравнений.
\end{remark}

\begin{definition}
    $A \in M_{m \times n}(R), B \in M_{n \times k}(R) \Longrightarrow A \cdot B = A \cdot 
    \large
    \begin{pmatrix}
        \begin{pmatrix}
            x_1
        \end{pmatrix}
        \dots
        \begin{pmatrix}
            x_n
        \end{pmatrix}
    \end{pmatrix} = \\\\
    \normalsize = 
    \begin{pmatrix}
        A \cdot x_1 \,\, \vrule \,\, ... \,\, \vrule \,\, A \cdot x_n 
    \end{pmatrix} = C \in M_{m \times k}(R)
    $ или более явно $C_{i, j} = \sum_{k = 1}^n A_{i,k} \cdot B_{k,j}$
\end{definition}

\quad
\begin{remark}
    $\forall A, B \not \Rightarrow A \cdot B = B \cdot A$\\
\end{remark}


\begin{examples} \\
    \begin{enumerate}
        \item $\begin{pmatrix}
            1&2\\
            3&4
        \end{pmatrix} \cdot \begin{pmatrix}
            0&1\\
            1&0
        \end{pmatrix} = \begin{pmatrix}
            2&1\\
            4&3
        \end{pmatrix}$, но $\begin{pmatrix}
            0&1\\
            1&0
        \end{pmatrix} \cdot \begin{pmatrix}
            1&2\\
            3&4
        \end{pmatrix} = \begin{pmatrix}
            3&4\\
            1&2
        \end{pmatrix}$ \\
        \item $Ax = b$ -- как уже говорилось, удобная запись системы линейных уравнений \\
        \item $a  = \begin{pmatrix}
            a_1\\
            \cdots\\
            a_n
        \end{pmatrix} \rightsquigarrow a^T = (a_1, \cdots, a_n)$ , тогда $a^T \cdot x$ -- скалярное произведение векторов $a, x$ \\
        \item $G$ - граф $\rightsquigarrow A(G)$ -- матрица смежности. Граф с петлями, ориентированный, (с кратными ребрами)\\\\ 
        $A(G)_{i,j} = \begin{cases}
            1, \,\text{если есть ребро из } i \,\text{в}\, j \, \text {(если граф с кратными ребрами, то меняем на количество)} \\ 
            0, \,\text{если ребра нет}
          \end{cases}$\\\\\\
          \textbf{Задача}: посчитать число путей из вершины $i$ в вершину $j$ длины $k$ -- $a_{i,j}^k$ \\\\
          \textbf{Решение}: 
              $k = 1 : a_{i,j}^1 = A(G)_{i,j};\,\, a_{i, j}^k = (A(G)) ^ k_{i, j}$, т.к $a_{i,j}^k = \sum_{l = 1}^n a_{i, l}^{(k-1)} \cdot a_{l, j}$ \\
        \item  $G$ - ориентированный граф, $W(e) $ -- вероятность переезда человека по ребру $e$, \\
        $W(L) = \prod_{e\in L}W(e)$ -- вероятность проехать по такому пути.

        $A(G)_{i, j} = W_{i, j}$ (вес ребра из $i$ в $j$) $\Longrightarrow$ по аналогии $A(G)^k$ - матрица вероятностей после $k$ переездов. 
          
        Общая интерпретация \textit{(Маркоская цепь)} : Сумма весов всех путей длины $k$. 
       

    \end{enumerate}
\end{examples}

\newpage
\subsection{Свойства произведения}

\begin{enumerate}
    \item Ассоциативность: $(AB)C = A(BC)$, где $A \in M_{m \times n}, B \in M_{n \times l}, C \in M_{l \times k} $\\
    \item $E_n = \begin{pmatrix}
        1 & ... &0\\
        &\rotatebox{135}{$\cdots$}&\\
        0&...&1 
    \end{pmatrix}$ --- единичная матрица \\\\\\
    $A \in M_{m\times n}, E_m \cdot A = A = A * E_n$
    \item Обратный элемент есть не всегда!\\\\
    \begin{example}
        $\begin{pmatrix}
            0&0\\
            0&1
        \end{pmatrix} \cdot 
        \begin{pmatrix} 
            a_1& a_2 & \cdots & a_n \\
            b_1& b_2 & \cdots & b_n 
        \end{pmatrix} = \begin{pmatrix}
            0&0&\cdots&0 \\
            b_1 & b_2 & \cdots & b_n 
        \end{pmatrix} \not = E$
    \end{example}
\end{enumerate}

\newpage
\subsection{Линейность}

\begin{remark}
    Логично предположить, что $\begin{pmatrix}
        a_1\\
        \cdots\\
        a_n
    \end{pmatrix} + \begin{pmatrix}
        b_1\\
        \cdots\\
        b_n
    \end{pmatrix} = \begin{pmatrix}
        a_1 + b_1\\
        \cdots\\
        a_n + b_n
    \end{pmatrix}$
\end{remark}

\quad

\begin{definition}
    $A,B\in M_{m \times n}(R) \Longrightarrow (A+B)_{i,j} = A_{i,j}+B_{i,j}$
\end{definition}

\quad

\begin{properties} 
    \item $A+B = B+A$ 
    \item $(A + B) + C = A + (B + C)$
    \item $0_{m \times n} = \begin{pmatrix}
        0&\cdots&0\\
        &\rotatebox{135}{$\cdots$}\\
        0&\cdots&0
    \end{pmatrix} : 0 + A = A + 0 = A$ 
    \item $(-A)_{i,j} = -A_{i, j}\,;\,\, A + (-A) = -A + A = 0$
    \item $(A + B) \cdot C = A C + B C$ и $C ( A + B) = CA + CB$
\end{properties}

\quad

\begin{remark}
    $\lambda (A B) = (\lambda A)B = A (\lambda B)$
\end{remark}

\quad

\begin{remark}
    $
\begin{cases}
    A \cdot x = b \\
    A \cdot y = c
\end{cases} \Longrightarrow A \cdot (x + y) = b + c
$
\end{remark}

\newpage 
\begin{remark} Решения систем уравнений похожи на диафантовые уравнения:\\

\quad\quad\quad\quad\quad\,\,\,\,\,    $
\begin{cases}
    A \cdot x = b \\
    A \cdot x_0 = 0 \text{ (однородная)}
\end{cases} \Longrightarrow A \cdot (x + x_0) = b
$     --- новое решение

\quad\quad\quad\quad\quad\,\,\,\,\,
$
\begin{cases}
    A \cdot x = b \\
    A \cdot y = b 
\end{cases} \quad \quad\quad\quad\quad\quad\,\,\, \Longrightarrow A \cdot (x - y) = 0
$     
\\

\quad\quad\quad\quad\quad\,\,\,\,\, Тогда можно попытаться угадать какой-то $x$ и свести задачу к поиску $x_0$
\end{remark}


\begin{remark} 
    Рассмотрим $X = \{ x \, | \, A \cdot x = 0 \}$ и $x_1, x_2 \in X \Longrightarrow x_1 + x_2 \in X, -x_1 \in X $ и $\lambda \cdot x_1 \in X$. Это похоже на определение подгруппы или идела в $R^n$.
\end{remark}

\newpage
\subsection{Векторные пространства}

\begin{definition}
    
    \quad $V = K^n, K$ - поле и операции:
    \begin{enumerate}
        \item $ (\lambda, v) \in K \times V \mapsto \lambda v \in V$ -- умножение на скаляр
        \item $(v_1, v_2) \in V \times V \mapsto v_1 + v_2 \in V$
    \end{enumerate} 
    \quad А также следующие аксиомы:
    \begin{enumerate}
        \item $(V, +)$ --- абелева группа
        \item $\lambda \cdot (v_1 + v_2) = \lambda v_1 + \lambda v_2$
        \item $(\lambda_1 + \lambda_2) \cdot v = \lambda_1 v + \lambda_2 v$
        \item $1 \cdot v = v$
        \item $(\lambda \mu)\cdot v = \lambda (\mu v)$
    \end{enumerate}
\quad Векторное (линейное) пространство над полем K --- $(V, +, \cdot) / K$
\end{definition}

\quad 

\begin{examples}
    \begin{enumerate}
        \item $K^n$ --- векторое пространство над над K 
        \item $M_{m \times n}(K)$ --- векторное над K
        \item $\mathbb{C}$ над $\mathbb{R}$ или более общая ситуация $K \subseteq L$ (подполе), тогда  $L$ --- векторное над $K$
        \item $c([a,b]) / \mathbb{R}$
        \item $c'([a,b]) \subseteq c([a,b]) / \mathbb{R}$
    \end{enumerate}
    
\end{examples}

\newpage
\begin{definition}
    Подпространство $W \subseteq V$ --- в.п. над $K$, если
    \begin{enumerate}
        \item $v, u \in W \Longrightarrow (u + v) \in W$
        \item $\lambda \in K, v \in W \Longrightarrow \lambda v \in W$
        \item $0 \in W$ (т.к. иначе $W = \emptyset$ подходит)
    \end{enumerate}
\end{definition}

\quad

\begin{example}
    $\{ x \,| \, A \cdot x = 0 \} $ --- подпространство в $K ^ n$
\end{example}



\begin{definition}
    $\lambda_1, ..., \lambda_n \in K; v_1, v_2, ..., v_n \in V $ -- в.п. $\Longrightarrow \lambda_1 v_1 + ... + \lambda_n v_n$ -- линейная комбинация $v_1, v_2, ..., v_n$ с коэффициентами $\lambda_1, ..., \lambda_n$ 
\end{definition}

\begin{motivation} 

    \quad\, $V = \mathbb{R} ^ 3$
    \begin{itemize}
        \item 3 вектора не лежат в одной плоскости $\Longleftrightarrow v = v_1\lambda_1 + v_2\lambda_2 + v_3\lambda_3$ 
        \item 3 вектора лежат в одной плоскости $\Longleftrightarrow v1 = v_2\lambda_2 + v_3\lambda_3 \Longleftrightarrow v_1\lambda_1 + v_2\lambda_2 + v_3\lambda_3 = 0$, где хотя бы один $\lambda$ не 0 
    \end{itemize} 
    
\end{motivation}


\begin{definition}
    $v_1, v_2, ..., v_n \in V$, тогда $v_1, v_2, ..., v_n$ --- линейно зависимые, если \\ $\exists \lambda_1, \lambda_2, ..., \lambda_n \in K : \lambda_1 v_1 + ... + \lambda_n v_n = 0$ и не все $\lambda_i = 0$
\end{definition}

\begin{definition}
    $v_1, v_2, ..., v_n \in V$, тогда $v_1, v_2, ..., v_n$ --- линейно независимые, если \\ $\nexists \lambda_1, \lambda_2, ..., \lambda_n \in K : \lambda_1 v_1 + ... + \lambda_n v_n = 0$ и не все $\lambda_i = 0$
\end{definition}