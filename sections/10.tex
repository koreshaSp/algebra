\section{Лекция 16.04}
\subsection{Жорданова клетка}
\begin{motivation}
    В прошлый раз мы пользовались тем, что алгебраическая кратность равна геометрической. Хотим понять в каких случаях это условие
    выполняется. Спойлер: не везде выполнено, но если расширить пространство до алгебраического замыкания(например $\R$ заменить на $\C$),
    то это условие будет выполнено всегда.
\end{motivation}
\begin{example}
    Пример, когда алгебраическая кратность не равна геометрической.
     \[
         A = 
         \begin{pmatrix}
             0 & 1\\
             0 & 0
         \end{pmatrix}
    .\] 
    $\chi_A(t) = t^2$, причём 
    $\ker A - 0 E\ker A = \langle \begin{psmallmatrix} 1\\0 \end{psmallmatrix}\rangle$
    видно, что в этой ситуации не может быть базиса из собственных векторов.
\end{example}
\begin{motivation}
    Хотим понять и формализовать все препятствия, которые возникают при диагонализации 
    матриц.
\end{motivation}
\begin{definition}
    $A$~--- Жорданова клетка размера $k$ для собственного числа $\lambda$, если
     \[
         A = 
         \begin{pNiceMatrix}
             \lambda & 1 & 0 & \Cdots & 0\\
             0 & \Ddots & \Ddots & \Ddots & \Vdots\\
             \Vdotsfor{2}&\Ddots& & & 0\\
                 & & & & 1\\
             0 & \Hdotsfor{2} & 0 & \lambda\\
         \end{pNiceMatrix} \in M_k
    .\] 
    Заметим, что $\chi_A(t) = (-1)^k(t-\lambda)^k= (\lambda-t)^k$,  $\dim \ker(A - \lambda E) = 1$
\end{definition}
\begin{remark}
    Матрицы такого вида показывают нам, что существуют матрицы, для которых алгебраическая
    кратность является произвольным числом, но геометрическая кратность при этом 1.
\end{remark}
\subsection{Инвариантные подпространства}
\begin{definition}
    $U\le V$ будем называть инвариантным относительно оператора $A$ если $A(U) \subseteq U$.
\end{definition}
\begin{remark}
    Если $v$~--- собственный вектор для $A$, то $\langle v \rangle$~--- инвариантное подпространство.
    $\dim U = 1$ и $U$~--- инвариантное, тогда $U = \langle v\rangle$, где  $v$~--- собственный для $A$.
    Поэтому инвариантные подпространства можно воспринимать как некоторое 
    обобщение собственного вектора.
\end{remark}
\begin{examples}
    \begin{enumerate}
        \item
            $\ker A$~--- инвариантное относительно $A$.
        \item
             $\im A$~--- инвариантное относительно $A$.
         \item
             $p(x)\in K[x]$, Договоримся, что в этот многочлен можно подставлять и матрицу
             следующим образом: $p(A) = a_0 + a_1 A + a_2 A^2 + \dots a_n A^n$.
             Заметим, что в этом случае есть очень удобное свойство:
             $(p(x)q(x))|_{x = A} = p(A)q(A)$. Тогда можно доказать, что $\ker p(A)$ является
             инвариантным подпространством относительно оператора $A$. Действительно, нам необходимо
             проверить, что $Ax \in \ker p(A) \Leftrightarrow p(A)(Ax) = 0$, если известно, что $p(A)x = 0$.
             Заметим, что матрица коммутирует с многочленом от неё же,
             то есть $p(A)(Ax) = (p(A)A)x = A\underbrace{p(A)x}_{=0} = 0$, что и требовалось
             доказать.
     \end{enumerate}
\end{examples}
\begin{remark}
    Поймём пользу которую нам может принести найденное инвариантное подпространство.
    Пусть $\langle v_1,\dots, v_k\rangle = U$ инвариантно относительно $A$. 
    Если \hyperref[thm:О дополнении до базиса]{дополнить этот набор до базиса}, получим 
    $v_1,\dots, v_k, v_{k + 1},\dots v_n$~--- базис $V$.
    Если каждый столбец $i$ соответствует вектору $v_i$, то получаем следующий вид матрицы $A$
    в базисе $v$.
    \[
        [A]^v_v = 
        \left(\begin{array}{c|c}
                A|_U & *\\
                \hline
                0 & A|_{V/U}
        \end{array}\right)
    .\] 
    Нули стоят в левой нижней части, так как мы знаем, что $A v_i \in U, \forall i\in \overline{1,k}$,
    а значит такие вектора выражаются через $v_1,\dots,v_k$.
    Что такое $V/U$ мы не обсуждали, но нам это не очень понадобится, поэтому можно считать, что в правой нижней части стоит что
    угодно. 
    
    Важно отметить то, что правая верхняя часть не влияет на характеристический многочлен.
    А это значит, что при помощи этого факта можно считать характеристический многочлен методом
    разделяй и властвуй, при условии что мы быстро умеем находить инвариантные подпространства
    размерности примерно $1/2$ от исходной.
\end{remark}
\begin{remark}
    В специфическом случае
    $V = U_1 \oplus U_2$. Прямая сумма инвариантных подпространств. Прошлое замечание приобретает вид:
     \[
         \left(\begin{array}{c|c}
                 A|_{U_1} & 0\\
                 \hline
                 0 & A|_{U_2}
         \end{array}\right)
    .\] 
\end{remark}
\begin{theorem}
    $A$~--- оператор на пространстве $V$ над полем $K$. $p(x), q(x) \in K[x]$~--- многочлены, для которых выполнено: 
     \begin{enumerate}
         \item $pq(A) = 0$ (свойство называется аннуляция оператора).
         \item $(p, q) = 1$ (многочлены взаимно простые).
    \end{enumerate}
    Тогда в этом случае $V = \ker p(A) \oplus \ker q(A)$.
\end{theorem}
\begin{proof}
    То, что $(p,q) = 1$ значит, что 
    $h(x)p(x) + g(x)q(x) = 1$(так как существует линейное разложение их НОД).
    А из этого можно получить следующее тождество, если переписать это на языке операторов:
    $h(A)p(A) + g(A) q(A) = E$, где $E$ является тождественным оператором, и равенство
    стоит между двумя операторами.
    Тогда можно подействовать этими эквивалентными операторами на вектор $v$, получить
    следующее равенство: $h(A)(p(A)v) + g(A)(q(A)v) = v$.

    Докажем, что $p(A)v\in \ker q(A)$ следует из $pq(A) = 0$.
    \[
        p(A)v\in \ker q(A) \Leftrightarrow q(A)(p(A)v) = 0 \Leftrightarrow (pq)(A)v = 0 \Leftarrow pq(A)=0
    \] 
    
    Получили, что $p(A)v\in \ker q(A), q(A)v\in \ker p(A)$ так как $pq(A)=0$,
    тогда в равенстве выше вектор $v$ разложился на сумму векторов из $\ker q(A), \ker p(A)$:
    \[
        v = 
        \overbrace{h(A)(\underbrace{p(A)v}_{\in \ker q(A)})}^{\in \ker q(A)} + 
        \overbrace{g(A)(\underbrace{q(A)v}_{\in \ker p(A)})}^{\in \ker p(A)}
    .\]
    Тут мы неявно
    воспользовались тем, что ядро многочлена оператора инвариантно оператору, а
    значит и многочлену от оператора.

    Теперь давайте докажем, что такое разложение $v$ единственно, для этого достаточно
    показать, что $\ker p(A) \cap \ker q(A) = 0$. Возьмём произвольный
    $x\in \ker p(A)\cap \ker q(A)$. Хотим показать, что $x = 0$.
    Из того, где лежит  $x$ следует, что $p(A)x = 0 = q(A)x$.
    Тогда в равенстве $h(A)(p(A)x) + g(A)(q(A)x) = x$ с обоих сторон находятся нули, 
    значит $x = 0$.
\end{proof}
Будем считать, что сейчас и далее работаем над алгебраически замкнутым полем $K$.
Посмотрим на набор операторов: $E, A, A^2, \dots$ над векторным пространством $V$.
Размерность пространства операторов конечномерно(и равно $(\dim V)^2$ так как
оператор \hyperref[thm:Линейное отображение однозначно задаётся двумя базисами]
{определяется однозначно двумя наборами базисов}), поэтому
с какого-то момента у нас появится линейная зависимость данного набора операторов.
А значит существует многочлен, который аннулирует $A$, назовём его $g$.
Так как $K$ алгебраически замкнуто, то у этого многочлена есть разложение на простейшие
$g(x) = \prod (x - \lambda_i)^{\alpha_i}$, а значит исходное пространство раскладывается как 
%todo: почему, как этот g вообще связан с пространством
\[
    V = \ker (A - \lambda_1 E)^{\alpha_1} \oplus \dots \oplus \ker (A - \lambda_k E)^{\alpha_k}
.\]
\begin{motivation}
    Хотим понять, как устроено действие ядра $A$ на подпространстве $\ker (A - \lambda_iE)^{\alpha_i}$.
    Иначе говоря к какому виду можно привести матрицу оператора $A$ в этом случае.
\end{motivation}
\begin{remark}
    $A - \lambda_iE$ на пространстве $\ker(A - \lambda_iE)^{\alpha_i}$ является нильпотентом.
    Так как $(A - \lambda_iE)^{\alpha_i} = 0$
\end{remark}
\begin{motivation}
    Таким образом вся классификация сводится к классификации нильпотентных элементов.
\end{motivation}
\begin{definition}
    %TODO Перенести в definition
    $J_k(\lambda)$~--- жорданова матрица.
    $J_k(\lambda) - \lambda E$~--- нильпотент.
\end{definition}
\begin{theorem}
    $A\colon V\mapsto V$. $K$~--- алгебраически замкнуто, $\exists$ базис $e$.
    Тогда 
     \[
         [A]^e_e =
         \begin{pmatrix}
             J_k(\lambda_1) & \dots & 0\\
             \vdots & \ddots & \vdots\\
             0 & \dots & J_k(\lambda_n)
         \end{pmatrix}
    .\] 
    \begin{definition}
        Такой базис $e$ будем называть Жордановым базисом.
    \end{definition}
    Такой вид матрицы $[A]^e_e$ единственный с точностью до перестановки блоков.
    Иначе говоря, какие бы два Жордановых базиса мы не взяли, то жорданова форма будет одинакова с точностью до перестановки блоков.
    %TODO: есть переформулировка данной теоремы.
\end{theorem}
\begin{proof}
    Докажем единственность, а существование сказали посмотреть в конспекте).
    Посмотрим какие числа могут стоять на диагонали. Очевидно, что это всегда собственные числа $\lambda_i$ матрицы $A$.
    $\chi_A(t) = \chi_J(t) = \pm\prod\limits_{}^{}{(t-\lambda_i)^{k_i}}$.

    Знаем, что у каждого $\lambda_i$ есть алгебраическая кратность. Заметим, что эта величина не зависит от выбора базиса.
    Алгебраическая кратность каждого $\lambda_i$ равна следующему: $\sum\limits_{\lambda_j=\lambda_i}^{}{k_j}$~--- сумма размеров
    клеток для $\lambda_i$. Геометрическая кратность $\lambda_i$  равна 
    $\dim \ker [A- \lambda_iE]^e_e = \dim \ker (J - \lambda_iE)$, что равно количеству клеток с собственным числом $\lambda_i$.

    $[\dim\ker(A - \lambda_iE)^{\alpha_i}]$, хотим понять можно ли по этому набору восстановить ${K_s}$ для 
    фиксированного $\lambda_i$.
\end{proof}
