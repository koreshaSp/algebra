\section{Лекция 11.04}
\begin{motivation}
    В прошлый раз мы пользовались линейным оператором (линейное отображение из пространства в себя),
    чтобы определить понятие ориентации. Теперь посмотрим, в каких ситуациях естественно,
    что линейное отображение принимает объект какого-то фиксированного типа и выдает объект этого же фиксированного типа. $V\xrightarrow{L}V$
\end{motivation}
\begin{examples}
    \begin{enumerate}
        \item Преобразование плоскости/пространства, повороты.
        \item $K[x]_{\le n} \mapsto K[x]_{\le n}$, где  $f\rightarrow f'$
        \item Марковские цепи. \\
        \begin{definition}
            \label{def:Марков}
            Цепь Маркова — последовательность случайных событий с конечным или счётным числом исходов,
            где вероятность наступления каждого события зависит только от состояния, достигнутого в предыдущем событии.
        \end{definition}
        \item Эволюция во времени.\\
            Пусть есть популяция, которая разделена по возрастам. Пусть $n_1, n_2,\dots, n_k$~--- количество особей каждого
			возраста. Пусть у нас заданы коэффициенты выживаемости($s_i$), которые показывают, что особь возраста $n_i$ проживёт
            сколько надо и станет особью возраста $n_{i+1}$. Пусть $f_i$ количество потомков, которое даёт в среднем особь возраста $n_i$.
            $s_i$ коэффициент выживаемости. 
            Хотим понять, как будет выглядеть матрица перехода количества особей с учётом времени (для фиксированных $f_i, s_i$).
            Утверждается, что так($k = 4$):
            \[
            \begin{pmatrix}
                f_1&f_2&f_3&f_4\\
                s_1&0&0&0\\
                0&s_2&0&0\\
                0&0&s_3&0\\
            \end{pmatrix}
            \begin{pmatrix}
                n_1\\
                n_2\\
                n_3\\
                n_4\\
            \end{pmatrix}
             =
             \begin{pmatrix}
                 n'_1\\
                 n'_2\\
                 n'_3\\
                 n'_4
             \end{pmatrix}
            .\] 
            Эта конструкция называется моделью Лесли.
            \begin{remark}
                Модель Лесли - это не Марковская цепь, хотя и очень похожа.
            \end{remark}
        \item
            Различные операторы в физике. Например, оператор Лапласа $\delta f$
        \item
            Линейные рекуррентные соотношения.\\
            $x_{n + k} = a_{k-1}x_{n + (k - 1)} + \dots + a_0x_n$.
            Тогда:
            $X_{n + 1} = AX_n$, где матрицу $A$ можно составить из коэффициентов $a_{k - 1}, \dots, a_0$, 
            $X_n = \begin{psmallmatrix}
                x_n \\ \vdots \\ x_{n + k - 1}
            \end{psmallmatrix}$
    \end{enumerate}
\end{examples}
\subsection{Собственные числа и вектора}
\begin{motivation}
    Давайте посмотрим на одномерный случай. Тогда линейный оператор - это просто умножение на какую-то константу.
    В самом деле всё элементарно $Av = \lambda\cdot v$, где $\lambda\in K$.
    Тогда легко видеть, что $A^nv = \lambda^n v$. Хотим получить что-то похожее, но в пространствах больших размерностей.
\end{motivation}
\begin{definition}
    $L\colon V\rightarrow V, \lambda\in K, v\not= 0\in V$. Тогда $\lambda$~--- собственное число $L$, а $v$~--- собственный вектор $L$,
    если $Lv = \lambda v$.
\end{definition}
\begin{motivation}
    Почему это понятие удобно. Пусть $v$ какое-то распределение на вершинах графа, на котором есть случайное блуждание.
    Хотим вычислить $P^n v$. $v = c_1v_1 + \dots + c_kv_k$, где все $v_i$~--- собственные вектора $P$,
    которым соответствуют собственные числа $\lambda_i$. Утверждается, что $P^n v = \sum c_i \lambda_i^n v_i$
    Получили существование очень удобного базиса для умножения на фиксированную матрицу в произвольной степени.
\end{motivation}
\begin{remark}
    Заметим, что такое возможно не всегда.
    Пусть 
    \[
    A = 
    \begin{pmatrix}
        0& 1\\
        -1&0\\
    \end{pmatrix}
    \text{~--- матрица поворота на~} 270^\circ 
    \]
    Мы хотим, чтобы после применения преобразования к собственному вектору, этот вектор сохранил направление.
    Но понятно, что для матрицы $A$ такого не может быть.
\end{remark}
\begin{motivation}
    Давайте научимся искать собственные числа и вектора.
\end{motivation}
\begin{remark}
    Напомним, что называется \hyperref[def:определитель отображения]{определителем линейного отображения}.
\end{remark}
\begin{statement}[Критерий для собственного числа]
    $\lambda$~--- собственное число оператора $L \Leftrightarrow \det(L - \lambda E) = 0$
\end{statement}
\begin{proof}
    $\exists v\not= 0\colon Lv = \lambda v$, что эквивалентно $\exists v\not=0\colon (L - \lambda E)v = 0$,
    это значит, что $v \in \ker (L \lambda E)$, то есть $\dim \ker (L - \lambda E) \neq 0$,
    а это то же самое, что $\det(L - \lambda E) = 0$
\end{proof}
\subsection{Характеристический многочлен и след матрицы}
\begin{definition}
    $L\colon V\rightarrow V$. Тогда характеристическим многочленом
    оператора $L$ называется следующее число:
    \[
        \chi_L(t) = \det(L - tE) = 
        \det \begin{pmatrix}
            a_{1,1}-t&a_{1,2}&\dots&a_{1,n}\\
            a_{2,1} & \ddots & \ddots& \vdots\\
            \vdots & \ddots & \ddots& a_{n-1,n}\\
            a_{n,1} & \dots & a_{n,n-1}& a_{n,n}-t\\
        \end{pmatrix}
    .\] 
\end{definition}
\begin{remark}
    Старшим коэффициентом $\chi_L(t), L\in M_n$ является $(-1)^n$.
    Это легко видеть, так как максимальная степень $t$, которая может получиться в выражении 
    определителя, берется из произведения элементов на главной диагонали.
\end{remark}
\begin{remark}
    \label{rem:Характеристический многочлен не зависит от базиса}
    $[L - tE]^e_e \in M_n\left(K[t]\right)$. Так удобно говорить,
    чтобы не зависеть от введённого базиса $e$ 
    (было \hyperref[def:определитель отображения]{доказано ранее}, что 
    $\det [L - t E]^e_e$ не зависит от выбора базиса).
    Из этого следует, что $\chi_L(t)$ действительно многочлен над полем $K[t]$ и не зависит
    от выбора базиса в котором записано $L$.
\end{remark}
\begin{definition}
    $A\in M_n(K)$. Сумма диагональных элементов называется следом матрицы $A$ и обозначается $\tr A$.
\end{definition}
\begin{remark}
    Свободный член $\chi_L(t)$ это $\det L$, а коэффициент при $t^{n-1}$ это $(-1)^{n-1} \tr L$.
\end{remark}
\begin{proof} 
    Сначала покажем, что свободный член $\chi_L(t) = \det L$. \\ $\chi_L(t) = \det (L - t E)$,
    а свободный член $\chi_L(t) = \chi_L(0) = \det (L - 0) = \det L$. \\
    Теперь покажем, что коэффициент $\chi_L(t)$ при $t^{n - 1}$ равен $(-1)^{n - 1} \tr L$. \\
    Вспомним формулу \hyperref[stm:Формула разложения]{разложения по столбцу} для определителя:
    $\det A = \sum\limits_\sigma \prod\limits_i a_{i\sigma(i)}$ и общую идею,
    что мы смотрим на все возможные расстановки ладей на матрице, чтобы они не били друг друга.
    Так как мы смотрим на коэффициент при $t^{n - 1}$, хотя бы $n - 1$ ладья должна стоять на диагонали,
    ну а так как матрица квадратная, мы не можем поставить последнюю ладью куда-то кроме последней свободной клетки на главной диагонали.
    Теперь понятно, что мы смотрим на выражение вида 
    $(a_{1 1} - t)\cdot(a_{2 2} - t)\cdot \dots \cdot (a_{n n} - t)$,
    из которого, раскрывая скобки, получим необходимое равенство.
\end{proof}
\begin{properties}
    \item $\tr (A + B) = \tr A + \tr B$
    \item След матрицы не зависит от выбора базиса.
    \item $\tr (AB) = \tr (BA)$, где $A \in M_{n \times m}(K), \; B \in M_{m \times n}(K)$
\end{properties}
\begin{proof}
    Первое свойство очевидно, второе будет следовать из доказанного ниже утверждения,
    третье проверяется просто расписав произведение матриц.
\end{proof}
\begin{statement}
    $\tr [A]^e_e = \tr [A]^f_f$, где $e,f$~--- некоторые базисы $V$.
\end{statement}
\begin{proof}
    $A$~--- линейный оператор, $C = [id]^e_f$~--- матрица перехода из стандартного базиса $e$ в какой-то базис $f$.
    Пользуясь \hyperref[]{формуле} выпишем следующее равенство:
    \[
        \tr [A]^e_e = 
        \tr [id]^f_e [A]^f_f [id]^e_f = 
        [\tr AB = \tr BA] =
        \tr [id]^f_e [id]^e_f [A]^f_f = 
        \tr [A]^f_f
    \] 
\end{proof}
\begin{example}
    \[
    A = 
    \begin{pmatrix}
        a&b\\
        c&d
    \end{pmatrix}
    .\] 
    Тогда $\chi_A(t) = t^2 - \tr A\cdot t + \det A$
\end{example}
\subsection{Диагонализуемость матриц}
\begin{motivation}
    Давайте подумаем существует ли ситуация, где мы можем рассчитывать на то, что любой вектор раскладывается в сумму собственных.
    Для этого нужно иметь базис из собственных векторов.
\end{motivation}
\begin{theorem}[О линейной независимости набора собственных векторов]
    $L: V\rightarrow V, v_1\dots, v_k,\\ \lambda_1, \dots, \lambda_k$~---
    собственные вектора и собственные числа $L$ соответственно, причём
    $\lambda_i\not= \lambda_j \forall i\not= j$.
    Тогда $v_1\dots v_k$ линейно независимые
\end{theorem}
\begin{proof}
    Докажем от противного. Предположим что они линейно зависимы. Тогда возьмём наименьшую возможную по количеству элементов
    нетривиальную линейную комбинацию равную 0(опускаем максимальное число слагаемых, равных нулю).
    $c_1v_1+\dots+c_k v_k = 0$. Давайте применим к правой и левой части $L$. Получим, следующее равенство
    $0 = c_1\lambda_1v_1+\dots + c_k\lambda_k v_k$. Не умаляя общности $c_1\not=0$ (так как есть ненулевое слагаемое).
    Теперь давайте сделаем следующий трюк: умножим исходное равенство на $\lambda_1$ и вычтем из текущего, тогда получим:
    \[
    0 = 0\cdot v_1 + (\lambda_1 - \lambda_2)c_2v_2 + \dots + (\lambda_1 - \lambda_k) c_k v_k
    .\]
    В этой сумме есть тоже ненулевое слагаемое, так как $\lambda_i \not= \lambda_j$ но это значит, что мы получили
    разложение нуля с меньшим количеством ненулевых слагаемых. Противоречие.
\end{proof}
\begin{follow}
    $L\colon V \mapsto V$, где $V$ векторное пространство над алгебраически замкнутым полем $K$(значит, что количество корней
    любого многочлена с учетом кратности равно его степени).
    Если все корни $\chi_L(t)$ различны, тогда существует базис из собственных векторов.
\end{follow}
\begin{proof}
    Пусть $\lambda_1\dots \lambda_n$~--- различные корни многочлена $\chi_L(t)$(их количество равно размерности пространства $n$).
    $v_1\dots v_n$~--- собственные вектора, соответствующие $\lambda_1\dots \lambda_n$,
    а значит линейно независимы по предыдущему следствию. Значит, $v_1\dots v_n$ --- базис 
\end{proof}
\begin{definition}
    $L\colon V \to V$~--- диагонализуема, если $\exists$ базис $f\colon [L]^f_f$~--- диагональная матрица.
    Если переписать на язык матриц, то получим: \\ $A\in M_n(K)$ --- диагонализуема $\Leftrightarrow \exists C\in GL_n(K) \colon C^{-1}AC$~---
    диагональная.
\end{definition}
\begin{definition}
    $L\colon V\mapsto V$, $\lambda$~--- собственное число. Тогда:
    \begin{enumerate}
        \item Алгебраическая кратность $\lambda$ --- 
            кратность  $\lambda$ как корня $\chi_L(t)$.
        \item Геометрическая кратность $\lambda$ ---
            $\dim \ker (L-\lambda E)$
    \end{enumerate}
\end{definition}
\lab{привести пример}
\begin{theorem}[Необходимое и достаточное условие диагонализуемости]
    $L\colon V\mapsto V$ 
    \begin{enumerate}
        \item $\chi_L(t)$~--- раскладывается на $n$ линейных множителей в $K$.
        \item алгебраическая кратность $\lambda$ = геометрическая кратность $\lambda$, $\forall\lambda\text{~--- собственное число}$ 
    \end{enumerate}
    Тогда и только тогда $L$~--- диагонализуемая.
\end{theorem}
\begin{proof}
    $L$~--- диагонализуема $\ora$ свойства.
    \begin{enumerate}
        \item 
            Понятно, что $\chi_L(t)$ и $\dim \ker L$ можно считать выбрав любой базис.
            Пользуясь тем, что $L$ диагонализуема, выберем $e$~--- 
            базис из собственных векторов $L$. 
            \[
              [L]^e_e = \begin{pNiceMatrix}
                \Block[borders={bottom,right,tikz=dashed}]{3-3}{}
                \lambda_1 & 0 & \Hdotsfor{7} & 0 \\
                0 & \Ddots & & & & & & & & \Vdotsfor{7}\\
                \Vdotsfor{7} & & \lambda_1 & & & & & & & \\
                 & & & \Block[borders={bottom,right,left,top,tikz=dashed}]{3-3}{} \lambda_2 & & & & & & \\
                 & & & & \Ddots & & & & & \\
                 & & & & & \lambda_2 & & & & \\
                 & & & & & & \ddots & & & \\
                 & & & & & & & \Block[borders={left,top,tikz=dashed}]{3-3}{} \lambda_n & & \\
                 & & & & & & & & \Ddots & 0\\
                0 & \Hdotsfor{7} & 0 & \lambda_n \\
              \end{pNiceMatrix}
              \]
            Мы выбираем такой порядок векторов в базисе $e$, чтобы базисные векторы,
            которые соответствуют одному и тому же собственному числу находились рядом. \\
            У такой матрицы удобно считать характеристический многочлен, его корнями будут просто элементы главной диагонали. 
            Алгебраическая кратность $\lambda_i \; \forall i$~--- это количество раз, которое $\lambda_i$ встречается на диагонали.
            Обозначим алгебраическую кратность $\lambda_i$ через $m_i$. \\
            Покажем, что геометрическая кратность $\lambda_i$ тоже равна $m_i$. Для $i = 1$ посмотрим на 
            \[
              [L]^e_e  - \lambda_1 E = \begin{pNiceMatrix}
                \Block[borders={bottom,right,tikz=dashed}]{3-3}{}
                0 & 0 & \Hdotsfor{7} & 0 \\
                0 & \Ddots & & & & & & & & \Vdotsfor{7}\\
                \Vdotsfor{7} & & 0 & & & & & & & \\
                 & & & \Block[borders={bottom,right,left,top,tikz=dashed}]{3-3}{} \lambda_2 - \lambda_1 & & & & & & \\
                 & & & & \Ddots & & & & & \\
                 & & & & & \lambda_2 - \lambda_1 & & & & \\
                 & & & & & & \ddots & & & \\
                 & & & & & & & \Block[borders={left,top,tikz=dashed}]{3-3}{} \lambda_n - \lambda_1 & & \\
                 & & & & & & & & \Ddots & 0\\
                0 & \Hdotsfor{7} & 0 & \lambda_n - \lambda_1\\
              \end{pNiceMatrix}
              \]
            Легко видеть, что $\lambda_i - \lambda_1 \neq 0 \; \forall i \neq 1$, а ядро - это подпространство, 
            натянутое на первые $m_1$ координат (мы просто обнулили первые $m_1$ диагональных элементов).
            Аналогично можно проделать для любого $i$.
    \end{enumerate}
    Теперь докажем $\ola$.
    
    Возьмём $\ker(L - \lambda_i E)$. Пусть $\dim \ker (L - \lambda_i E) = k_i$, $u_{1,i},\dots,u_{k_i,i}$~--- базис $\ker(L - \lambda_i E)$. 
    Благодаря условиям теоремы(знаем, что количество корней
    $\chi_L(t) = n$, а значит сумма алгебраических кратностей равна $n$, ну а сумма 
    геометрических кратностей равна им и тоже равна $n$) мы знаем, что $\sum k_i = n = \dim L$.
    Тогда на данный момент у нас есть $n$ векторов $u_{i,j}$, давайте проверим, что они
    образуют базис пространства $V$, если это так, то мы получили базис из собственных
    векторов.
    \begin{itemize}
        \item Линейная зависимость.\\
        \[
                \sum\limits_{i,j}c_{i,j}u_{i,j} = 0 \Leftrightarrow
                \sum\limits_{j}\sum\limits_{i}c_{i,j}u_{i,j} = 0
        .\] 
        заметим, что $\sum\limits_{i}c_{i,j}u_{i,j}$
        это линейная комбинация собственных векторов с собственным числом $\lambda_j$,
        иначе говоря это линейная комбинация векторов лежащих в $\ker (L - \lambda_i E)$,
        следовательно эта сумма является вектором из $\ker (L - \lambda_i E)$.
        Значит имеем следующую сумму:
        \[
            \sum\limits_{j}\underbrace{\sum\limits_{i}c_{i,j}u_{i,j}}_{\in \ker L - \lambda_i E} = 0
        \] 
        Отбросим все нулевые слагаемые внешней суммы.
        Тогда у нас останется сумма собственных векторов для различных собственных чисел,
        она линейно независима(по доказанной \hyperref[thm:О линейной независимости набора собственных векторов]
        {выше теореме о том, что набор собственных векторов линейно независим}),
        следовательно $\sum\limits_{i}^{}{c_{i,j} u_{i,j}} = 0, \forall j$,
        но так как, $u_{i,1},\dots, u_{i, k_i}$ является линейно независимым
        набором такое возможно только когда все $c_{i,j}$ равны нулю.
        \item Является образующей пространства $V$.\\
            Выполнено автоматически, так как мы имеем линейно независимый набор
            размера $n = \dim V$.
    \end{itemize}
\end{proof}
\begin{remark}
    Если $f(x_1,\dots, x_n)\not= 0\in \R[x_1\dots x_n]$, то вероятность попасть в множество нулей многочлена равна нулю
    $\pr [f(x_1,\dots, x_n)] = 0$.
\end{remark}
\begin{remark}
    $f(x) = a_0 + \dots + a_nx^n$ имеет два одинаковых комплексных корня равносильна тому, что 
     $\exists D(a_0,\dots, a_n) = 0$. $D$ называется дискриминантом. \\
     Дополнительный факт(упражнение, подсказка: определитель Вандермонда):
     \[
         D = a_n^{2n - 2}\prod\limits_{}^{}{(\lambda_i - \lambda_j)^2}
     \]
\end{remark}
\begin{motivation}
    Давайте поймем, что происходит в ситуации $A^n v$.
\end{motivation}
\begin{statement}[Ассимптотика диагонализуемого оператора]
    Пусть $A\in M_n(\C)$~--- диагонализуемая, $v\in \C^n$, тогда $A^kv = c_1\lambda_1^kv_1 +
    \mathcal{O}(\lambda_2^k)$($k\rightarrow \infty$), где $\abs{\lambda_1}\ge \abs{\lambda_2} \ge \dots \ge \abs{\lambda_n}$.
\end{statement}
\begin{proof}
    Очевидно, так как 
    \[
        A^k v = c_1\lambda_1^k v_1 + \dots + c_n \lambda_n^k v_n
    .\] а благодаря тому, что $\abs{\lambda_1} \ge \abs{\lambda_i}, \forall i \not= 1$ все члены, кроме первого могут быть
    записаны как $\mathcal{O}(\lambda_2^k)$.
\end{proof}
\begin{follow}
    \[
        \begin{gathered}
            \lambda_1 \sim \frac{A^{k + 1}v[1]}{A^k v[1]}\\
            \frac{A^kv}{\norm {A^k v}}\rightarrow cv_1
        \end{gathered}
    .\] 
    Таким образом мы можем считать приближённо значение собственных чисел.
\end{follow}
