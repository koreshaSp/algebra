\section{Лекция 04.04}
\subsection{Базовые свойства определителя}
\begin{theorem}[Свойства определителя]\leavevmode
    \begin{enumerate}
        \item $det A$~--- форма объёма на $K^n$.
        \item Все остальные формы объёма пропорциональны $det$.
        \item $det A = det A^T$.
        \item Элементарные преобразования строк и столбцов.
        \item  $det(AB) = detAdetB$.
        \item  \[
        \left(\begin{array}{c|c}
                A & B\\
                \hline
                0 & C
        \end{array}\right) = det A det C
        .\] 
    \item $det(A^{-1}) = (det A)^{-1}$
    \end{enumerate}
\end{theorem}
\begin{proof}\leavevmode
    \begin{enumerate}
        \item 
            Докажем то, что $det A$~--- форма объёма по определению.
            \begin{itemize}
                \item $det: (K^n)^n\mapsto K$(todo)
                \item Полилинейнойсть очевидна(todo)
                \item Кососимметричность проверяется следующим образом:
                    $det(\dots, v,\dots, v,\dots) = 0$.
                    Для этого матрицу $\big(\dots, v,\dots, v,\dots\big)$ обозначим за $A$.
                    Тогда можно расписать по определению:
                    $det A = \sum\limits_{\sigma\in S_n}^{}{sgn\sigma\prod\limits_{i=1}^{n}a_{i,\sigma(i)}}$.
                    Для получения группы всех нечетных перестановок из группы всех 
                    чётных можно воспользоваться следующим отображением $\tau=(kl)$,
                    где $k, l$~--- индексы векторов $v$ в матрице $A$. Тогда,
                    если вернуться к нашей формуле:
                    $det A = \sum\limits_{\sigma\in S_n}^{}{\prod\limits_{i=1}^{n}a_{i,\sigma(i)}}-
                    \sum\limits_{\sigma\in S_n}^{}{\prod\limits_{i=1}^{n}a_{i,\tau\sigma(i)}}$.
                    Заметим, что если $\sigma(i)\not= k,l$, тогда $a_{i, \tau\sigma(i)} = a_{i,\sigma(i)}$.
                    Значит куча слагаемых у нас сокращается и остаётся только 
                    позиции $l, k$. Но они тоже сократятся(можно проверить это ручками так
                    как слагаемых всего 2).
            \end{itemize}
        \item 
            Доказано в прошлый раз, так как все формы объёма определяются через определитель умноженный
            не некоторую константу.
        \item
            Доказываем $det A = det A^{T}$. Распишем по определению:
            $det A = \sum\limits_{\sigma\in S_n}^{}{sgn\sigma\prod\limits_{i=1}^{n}a_{i,\sigma(i)}}$.
            $det A^T = \sum\limits_{\tau\in S_n}^{}{sgn\tau\prod\limits_{i=1}^{n}a_{i,\tau(i)}}$.
            Заметим, что $sgnT = sgn T^{-1}$. Кроме этого 
            $\prod\limits_{i = 1}^{n}a_{\tau(i)i} = \prod\limits_{j=1}^{n}a_{j\tau^{-1}(j)}$.
        \item $L$~--- кососимметричная и полилинейная. 
            Тогда $L(\dots, u, \dots, v + \lambda u, \dots) = L(\dots, u, \dots, v,\dots) +
            \lambda L(\dots, u,\dots, u,\dots)$.
            Значит можно утверждать, что определитель не меняется при первом элементарном преобразовании
            строк/столбцов.
            Неизменность определителя для второго элементарного преобразования следуем из свойств 
            полилинейности/кососсимметричности.
            \begin{remark}
                Таким образом можно считать определитель при помощи метода Гаусса постепенно 
                упрощая нашу матрицу. 
                Иногда бывает удобно использовать и преобразования строк и преобразования столбцов.
            \end{remark}
        \item
            $det(AB) = det A det B$.
            Для доказательства этого свойства воспользуемся доказанным ранее(todo).
            Зададим следующую функцию $f(x) = det(AX)$. Хочется показать, что $f$~--- форма объёма
             \begin{itemize}
                 \item Кососимметричность. (помахали руками todo)
                 \item Полилинейность. Следует из полилинейности определителя.
                     $(AX) = A(u + v) = Au + Av$.
                     $det A, x = E_n$.
                     TODO(всё зафейлил)
             \end{itemize}
         \item
             Зададим форму объёма $f(X) = det \left(\begin{array}{c|c}
                     X & B\\
                     \hline
                     0 & C
             \end{array}\right) = det\left(\left(\begin{array}{c|c}
                     E_n & B\\
                     \hline
                     0 & C
             \end{array}\right)\right)detX$.
             Очевидно что $det\left(\left(\begin{array}{c|c}
                     E_n & B\\
                     \hline
                     0 & C
             \end{array}\right)\right)$ полилинейное по строчкам.
             Поэтому $det\left(\left(\begin{array}{c|c}
                     E_n & B\\
                     \hline
                     0 & C
             \end{array}\right)\right) = det C \cdot det\left(
                 \left(\begin{array}{c|c}
                         E_n & B\\
                         \hline
                         0 & E_m
                 \end{array}\right)
             \right)$. Можно заметить, что определитель этой матрицы равен одному,
            это легко понять, если поприменять элементарные преобразования.
            Значит исходная формула имеет вид:
             $f(X) = det C det X$. Что и требовалось доказать.
         \item
             $det A^{-1} = (det A)^{-1}$. $detA\cdot det A^{-1} = det(A\cdot A^{-1}) = det E_n = 1$
    \end{enumerate}
\end{proof}
\subsection{Связь определителя с объёмом}
\begin{motivation}
    Вернёмся к тому, зачем мы вообще ввели определитель.
\end{motivation}
\begin{statement}
    $\abs{det A} = Vol(\text{параллелепипед натянутый на столбцы $A$})$.
\end{statement}
\begin{proof}
    Потребуются свойства элементарных преобразований: 
    \begin{enumerate}
        \item $Vol(\dots, u, \dots, v+\lambda u,\dots) = Vol(\dots, u, \dots, v,\dots)$. (как следствие принципа Кавальери)
        \item $Vol(\dots, \lambda u,\dots) = \abs{\lambda} Vol(\dots,u,\dots)$.
        \item  $Vol(E_n) = 1$
    \end{enumerate}
    В совокупности эти 3 свойства позволяют нам вычислять объём параллелепипеда(todo).
    Они же позволяют вычислить модуль определителя. При этом надо заметить, что при
    одинаковых преобразованиях $\abs{det A}$ и $Vol()$ меняются на одинаковую величину.
\end{proof}
\subsection{Ориентация}
\begin{motivation}
    Поймём что даёт нам знание о знаке определителя.
\end{motivation}
\begin{definition}
    $V$~--- векторное пространство над  $\R$. $e_1,\dots, e_n$, $f_1,\dots, f_n$~--- базисы,
    тогда если $det[id]^f_e>0$ тогда базисы называются одинаково ориентированными.
\end{definition}
\begin{remark}
    На самом деле определение ориентированности задаёт классы эквивалентности на множестве
    базисов пространства.
\end{remark}
\begin{definition}
    $L\colon V\mapsto V$. Тогда $L$ называется оператором.
\end{definition}
\begin{motivation}
    Хотим понять, как оператор влияет на ориентацию базиса.
\end{motivation}
\begin{remark}
    Определитель матрицы линейного отображения $[id]^e_f$ не зависит от выбора базисов $e, f$.
\end{remark}
\begin{proof}
    $[L]^e_e = [id]^f_e [L]^f_f [id]^e_f$. $[id]^f_e = ([id]^e_f)^{-1}$, а значит 
    $det [id]^f_e = (det [id]^e_f)^{-1}$. Значит $[L]^e_e = [L]^f_f$(что и требовалось доказать)
\end{proof}
\begin{definition}
    $det [L]^e_f = det L > 0$. Тогда говорят, что $L$~--- не меняет ориентацию.
\end{definition}
\begin{remark}
    Упражнение: доказать, что $e_1,\dots, e_n$ и $Le_1,\dots, Le_n$ одинаково ориентированны.
\end{remark}
\begin{definition}
    $SL_n(K) = \left\{A\in M_n(K)\mid det A = 1 \right\}$
\end{definition}
\begin{theorem}[Формула разложения по строке или по столбцу]
    Будем говорить про столбцы, так как к строкам сводится в одно действие $det A = det A^T$.
\end{theorem}
\begin{proof}
    $A = C_1a_{1,1} + \dots + C_na_{n,1}$
    Тогда каждое $C_i$ можно найти как опредлитель матриц следующего вида
     \[
    \left(\begin{array}{c|ccc}
            1 & & * &\\
            \hline
            0 & &  &\\
            \vdots & & A_{\overline{1},\overline{1}} &\\
            0 & &  &\\
    \end{array}\right)
    .\] 
    todo
\end{proof}
\begin{definition}
    $A\in M_{m\times n}(K)$. $I \subseteq \{1,\dots, m\}; J\subseteq \{1,\dots, n\}$.
    $A_{I,J}\in M_{\abs{I}\abs{J}}(K)$.
    $A_{\overline{i}, \overline{j}} = A_{I,J}$, где 
    $I = \{1,\dots, m\}\setminus \{i\},
    J = \{1,\dots, n\}\setminus \{j\}$.
\end{definition}
\begin{statement}
    $det A = \sum\limits_{i=1}^{n}{(-1)^{i+j}det(A_{\overline{i},\overline{j}})a_{i,j}}, \forall j$
    (TODO?)
\end{statement}
\begin{proof}
    todo тут вообще всё запуталось(весь subsection)
\end{proof}
\begin{follow}
    $Ax = b$, $A$~--- обратимая.
    $x_i = \frac{\delta_i}{\delta}$, где $\delta = det A$, 
    \[
    \delta_i = \left(\begin{array}{c|c|c}
             & & \\
            \dots & b & \dots \\
             & & \\
    \end{array}\right)
    .\] 
\end{follow}
\begin{proof}
    $\delta_i = det(\dots | \sum\limits_{}^{}{x_jv_j} | \dots)$ 
    todo()
\end{proof}
\begin{definition}
    Давайте определим $M_{i,j}$(минор матрицы $A$), следующим образом:
    $M_{i,j} = detA_{\overline{i}, \overline{j}}$, где $A_{\overline{i}, \overline{j}}$~--- матрица $A$, в которой 
    отсутствует $i$ строка и $j$ столбец.
\end{definition}
\begin{definition}
    Пусть $A$~--- матрица. Тогда её алгебраическое дополнение(обозначается $A^{i,j}$) в точке $i, j$ определим следующим
    образом $A^{i,j} = (-1)^{i + j}M_{i,j}$. 
\end{definition}
\begin{definition}
    Присоединённой матрицей(обозн. $Adj A$) назовём следующую матрицу:\\ $(Adj A)_{i,j} = A^{j,i}$.
\end{definition}
\begin{statement}
    $A^{-1} = \frac{AdjA}{det A}$
\end{statement}
\begin{proof}
    Упражнение.
\end{proof}
