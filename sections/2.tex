\section{Лекция 21.02}
\subsection{Свойства линейной зависимости}
\begin{definition}
    Для пространства $K^n$ следующий набор назовём стандартным
    базисом: $$
    \left\{
    \begin{pmatrix}
    1\\ 0\\ \vdots\\ 0\\ 0
    \end{pmatrix},
    \begin{pmatrix}
    0\\ 1\\ \vdots \\0 \\0
    \end{pmatrix}
    ,\dots,
    \begin{pmatrix}
    0\\ 0\\ \vdots \\1 \\0
    \end{pmatrix}
    \begin{pmatrix}
    0\\ 0\\ \vdots \\0 \\1
    \end{pmatrix}\right\}$$
\end{definition}
\begin{remark}
    Очевидно, что этот набор линейно независим.
    Точно так же работает для матриц, один элемент главной диагонали которых единица.
\end{remark}
\begin{remark}
    Линейная зависимость в случае двух векторов $\Leftrightarrow$ вектора пропорциональны.
\end{remark}
\begin{lemma}
    $v_1,\dots v_n$~--- набор линейно независимых векторов, тогда новый набор векторов
    $v_1, \dots v_i, v_j + \lambda v_i, \dots v_n, i\not=j$ тоже линейно независим $\forall \lambda$.
\end{lemma}
\begin{proof}
    Обозначим новый набор векторов как $v'_i$.
    Рассмотрим $\sum_{k = 1}^{n}\mu_k v'_k = 0$, надо понять существуют ли такие нетривиальные $\mu_k$.

    Раскрыв скобки, получим:
    $(\sum_{k=1}^{n}\mu_kv_k) + \mu_j\lambda v_i = 0 \Rightarrow \mu_k=0\quad\forall k\not=i$ и $\mu_i + \mu_j \lambda = 0$
    по линейной независимости векторов(а значит коэффициенты при всех векторах точно 0).
    Значит $\mu_i=0$, так как $\mu_j\lambda= 0$.
\end{proof}
\begin{motivation}
    Доказав факт выше мы получили что-то похожее на элементарное преобразование системы первого типа.
    Теперь давайте будем двигаться в сторону понятия <<размерность>>.
\end{motivation}
\begin{theorem}{О линейной независимости линейной комбинации}
Для произвольного набора векторов
$v_1,\dots v_n\in V$, и $u_1,\dots, u_m\in V$, где каждый $u_i$~--- это линейная комбинация $v$. 
Тогда, если $m > n$, то набор векторов $u_1,\dots,u_m$ линейно зависим.
\end{theorem}
\begin{proof}
    Будем доказывать индукцией по $n,m$.

    База индукции:\\$n = 1, m > 2$, все $u_i = \lambda_i\cdot v_1$, тогда по факту выше $u$ линейно зависимы.

    Индукционный переход: $(n-1,m-1)\rightarrow (n,m)$:\\
    Определим $\lambda_{i,j}$ следующим образом: $u_i = \sum_{j=0}^{n} \lambda_{i,j} v_j$.
    Найдём вектор $u_k\colon \lambda_{k,n}\not=0$, если такого нет, тогда оказывается, что $v_n$ не
    участвует в разложении ни одного $u_i$, а значит можем воспользоваться предположением индукции.

    Перейдём к доказательству случая, где $u_k$ существует.
    Посмотрим на новый набор векторов $u' = \left\{u_i - \frac{\lambda_{i,n}}{\lambda_{k,n}}u_k \mid i\not=k\right\}$.
    Тогда каждый $u'_i$ выражается через $v_1,\dots,v_{n-1}$.
    То есть у нас есть $m-1$ вектор, каждый из которых выражается
    через $n-1$ вектор. А это значит, что данный набор $v'_1,\dots, v'_{m-1}$ линейно зависим по предположению индукции.

    Расписав по определению, получим $\sum_{i=0}^{m-1}\mu_i u'_i = 0$, где не все $\mu_i=0$.
    Распишем то же самое, но через $u_i$: $\sum_{i=0}^{m-1}\mu_iu_i + (\sum\dots)u_m=0$. Получается, что в этой
    линейной комбинации тоже не все коэффициенты равны нулю, а значит мы доказали линейную зависимость $u_1\dots u_m$. 
\end{proof}

\subsection{Базис}
\begin{definition}
    \textbf{Подпространством, порождённым набором векторов} $v_1, \dots, v_n\in V$, называется
    $$\{\lambda_1v_1 + \dots \lambda_nv_n \mid \lambda_i\in K\}\le V$$
    Далее обозначается как $\langle v_1,\dots, v_n\rangle$.\\
    \notice Причём это наименьшее подпространство содержащее $v_1,\dots,v_n$.
\end{definition}
\begin{definition}
    Если $\langle v_1,\dots v_n\rangle = V$, то набор векторов называется
    порождающей системой.
\end{definition}
\begin{definition}
    $v_1, \dots, v_n\in V$~--- базис, если:
    \begin{enumerate}
        \item $v_1,\dots,v_n$ линейно независимы
        \item $v_1,\dots, v_n$~--- порождающая $V$
    \end{enumerate}
\end{definition}
\begin{lemma}
    $v_1,\dots v_n \in V$ является базисом $\Leftrightarrow \forall v\in V \; \exists !\;
    \lambda_1,\dots,\lambda_n\colon\; v = \sum_{i=1}^{n}\lambda_iv_i$.
\end{lemma}
\begin{proof} \leavevmode\\
    $\ola$
    \begin{enumerate}
        \item Пусть набор $v$ линейно зависим, тогда: $\exists \mu \colon \sum_{i=1}^{n}\mu_iv_i=0, \exists \mu_j\not=0$.
            Тогда расписав любой вектор, воспользовавшись условием, получим 
            $v = \sum_{i=1}^{n}\lambda_iv_i = \sum_{i=1}^{n}(\lambda_i+\mu_i)v_i$. То есть мы нашли
            второе разложение $v$ как линейную комбинацию $v_i$, что противоречит с единственностью $\lambda$.
            Значит $v$ линейно независим.
        \item $v_1,\dots, v_n$~--- порождающая $V$ просто так как любой вектор из $V$ выражается
            линейной комбинацией векторов из $v$.
    \end{enumerate}
    $\ora$

    $\exists \lambda_i \sum_{i=1}^{n} \lambda_iv_i = V$
    Пусть есть две линейные комбинации, дающие $v$.
    $\sum_{i=1}^{n} \lambda_iv_i = \sum_{i=1}^{n} \mu_iv_i = v$. Можем вычесть одну из другой, тогда
    $\sum_{i=1}^{n}(\lambda_i - \mu_i)v_i = 0\Rightarrow \lambda_i-\mu_i=0\;\forall i
    \Rightarrow \lambda_i = \mu_i\;\forall i$.
\end{proof}

\begin{example}
    $1, x, x^2, \dots, x^n, \dots$~---~базис $K[x]$. 
    Проблема бесконечного базиса в том, что многие факты доказываются сильно сложнее.
    Поэтому полноценно работать с пространствами, которые имеют бесконечный базис, в рамках этого
    курса мы не будем, но иногда будем приводить примеры таковых.
\end{example}
\begin{theorem}{О дополнении до базиса}
    $u_1,\dots,u_m\in V$~--- порождающий набор в $V$. $v_1,\dots,v_n\in V$~---
    линейно независимый. Тогда $v_1,\dots, v_n$ можно дополнить до базиса добавив
    какие-то вектора из $u_1\dots, u_m$.
\end{theorem}
\begin{proof}
Идея банальна: добавляем по одному и в какой-то момент мы достигнем того, что набор
линейно независимым, но ни один вектор из $u$ мы не можем добавить, не сделав набор
линейно зависимым.

Посмотрим на количество векторов из $u_1,\dots, u_m$, которые не
лежат в линейной комбинации в пространстве, порождённом $\langle  v_1,\dots, v_n\rangle$
$v_1,\dots,v_n, u_1,\dots, u_k$ набор векторов в момент остановки. Хотим понять, что
мы получили базис.
Заметим, что в $\langle v_1,\dots,v_n,u_1,\dots,u_k\rangle$ нельзя добавить ни одно $u_i\; i > k$, а значит
любое $u_i$ выражается через вектора из нашего множества(очевидно). Тогда мы получили
следующую цепочку:
$$V\supseteq \langle v_1,\dots,v_n, u_1,\dots,u_k\rangle \supseteq \langle u_1,\dots,u_m\rangle=V$$
\end{proof}
\follow~$v_1,\dots,v_n$~--- пуст $\Rightarrow$ в любом векторном пространстве есть базис.
\subsection{Размерность}
\begin{theorem}{О размере базисов}
    $V = \langle u_1,\dots,u_m\rangle \Rightarrow$ размер любых двух базисов $V$ одинаков и конечен.
\end{theorem}
\begin{proof}
    $e_1,\dots,e_n$, $f_1,\dots,f_\alpha$~--- базисы. Знаем, что все $f_j$
    выражаются через $e_i$ и все $e_i$ выражаются через $f_j$.
    Допустим $\alpha > n$, тогда $f_1,\dots,f_\alpha$ линейно зависимы(по \ref{О линейной независимости линейной комбинации}), а значит $f$ не базис.
    Для случая $\alpha < n$ всё аналогично.
\end{proof}
\begin{definition}
    $V$~--- век. пространство $\dim V =$ количество векторов в базисе $V$.
\end{definition}
\begin{definition}
    $V$~--- конечномерное, если $V = \langle v_1,\dots, v_m\rangle$.
\end{definition}
\begin{lemma}
    Если $V$~--- конечномерное $U \leq V$, более того $\dim U \leq \dim V$, причём $\dim U = \dim V \Leftrightarrow U=V$.
\end{lemma}
\begin{proof}
    Возьмём $u_1,\dots,u_k$~--- линейно независимый набор из $U$, $k\leq \dim V$ так как этот же набор линейно независим и в $V$. 
    Получаем, что размер любой линейно независимого набора в $U$ не превосходит $\dim V$.

    Но для доказательства факта, необходимо доказать, что $U$~--- конечномерное пространство. Допустим, что это не так.
    Тогда возьмём некий линейно независимый набор $u_1,\dots, u_m$ и будем дополнять его элементами пространства $U$ так, чтобы набор оставался линейно независимым.
    Если $U$ не конечномерное пространство, то операцию выше можно повторять бесконечное число раз,
    а значит в какой-то момент $\dim\langle u_1, \dots, u_m\rangle$ станет больше $\dim V$. 
    Получили противоречие с $\langle u_1, \dots, u_m\rangle \subset U \subset V$.
\end{proof}
\subsection{Алгебраические и трансцендентные числа}
\begin{definition}
$\alpha\in\C, \alpha$~--- алгебраическое, если $\exists p(x)\not=0\in\Q[x] \colon p(\alpha)=0$.
Иначе $\alpha$~--- трансцендентное.
\end{definition}
Если $\alpha$~--- алгебраическое, $f(\alpha) = a_0 + a_1\alpha + \dots + a_n\alpha^n$
Вопрос: $f(\alpha)$~--- алгебраическое или нет?
Ответ:
$\alpha$~--- алгебраическое $\Rightarrow f(\alpha)$~--- алгебраическое.
\begin{proof}
    Определим векторное пространство над $\Q$, как $V = \langle1, \alpha, \alpha^2,\dots,\alpha^n,\dots\rangle = \langle1, \alpha, \alpha^2,\dots,\alpha^n\rangle \le \C$.
    Почему выполняется второе равенство?
    По алгебрaичности $\alpha$ $\exists p(\alpha) = b_0+b_1\alpha+\dots+b_n\alpha^n = 0, b_n\not=0$, тогда
    $\alpha^n = -\frac{b_0}{b_n} - \frac{b_1\alpha}{b_n} + \dots + \frac{b_{n-1}\alpha^{n-1}}{b_n}$.
    Обозначим $\beta = f(\alpha)$ и будем пытаться найти $q(\beta)=0$.
    Заметим, что $\beta\in V, \beta^2\in V,\dots,\beta^n\in V$. 
    Тогда найдём минимальную $k: \beta,\beta^2,\dots,\beta^k$~--- линейно зависимы, такой $k$ существует, так как $k < n$. Ну а линейная зависимость этого набора означает существование нетривиальных 
    $c_0,\dots, c_k\colon c_0 + c_1\beta + c_2\beta^2 + \dots + c_n\beta^n=0$.
\end{proof}
\subsection{Связь с теорией множеств}
\begin{definition}
    $U_1, U_2\leq V \Rightarrow U_1\cap U_2 \leq V$ и $U_1+U_2 =
    \{u_1+u_2\mid u_1\in U_1; u_2\in U_2\}\in V$
\end{definition}
\begin{theorem}{Формула Грассмана}
$V$~--- конечномерное. $U_1, U_2\leq V$. 
$\dim U_1 + \dim U_2 = \dim(U_1+U_2) + \dim(U_1\cap U_2)$
\end{theorem}
\begin{remark}
Для трёх подпространств и более она \textbf{не работает} по аналогии с формулой включений-исключений.
\end{remark}
