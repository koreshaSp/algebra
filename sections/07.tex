\section{Лекция 28.03}
\subsection{Определитель}
\begin{motivation}
    Хотим понять что мы хотим от объёма в общем случае и понять каким наборам
    аксиом он должен удовлетворять.
\end{motivation}
\begin{motivation}
    Давайте пока думать о частном случае, а именно: $\R^n$.
     $Vol(E_n)$~--- объём единичного куба. Давайте выпишем свойства которые мы в итоге хотим видеть.
     \begin{enumerate}
         \item $Vol(E_n) = 1$
         \item $Vol(v_1,\dots, \lambda v_i, \dots, v_n) = \abs{\lambda} Vol(v_1,\dots, v_i,\dots, v_n)$ 
         \item \lab{картинка 1} Давайте подумаем о случае $\R^3$ ищем площадь основания и 
             умножаем на высоту.  Давайте думать об одном векторе, как о сумме двух,
             параллельных плоскости, а оставшиеся два будут образовывать плоскость.(todo).\\
             $Vol(v_1,\dots, v_i + \hat{v_i}, \dots, v_n) = Vol(\dots, v_i, \dots) + Vol(\dots,\hat{v_i},\dots)$.
             Более того, мы хотим, чтобы $v_i, \hat{v_i}$ лежали по одну сторону подпространства 
             заданного остальными векторами.
         \item $Vol(v_1,\dots, v_i + \lambda v_j, \dots, v_j, \dots, v_n) 
             = Vol(v_1,\dots, v_i,\dots, v_j, \dots, v_n)$ 
             \lab{рисунок 2}
             \lab{принцип Кальвае чёто там про интегрирование при помощи прямой}
             \lab{рисунок 3}
         \item $Vol(v_1,\dots, v_i, \dots, v_i, \dots, v_n) = 0$ (на самом деле его можно вывести
             из предыдущих, но всё же это свойство важно).
     \end{enumerate}
     Наиболее интересными свойствами являются: 2,3,5.
\end{motivation}
Теперь мы готовы задать объём в случае произвольного поля.
\begin{definition}
    $Vol: V\times \dots \times V \mapsto K$, где $K$~--- поле, а $\dim V = n$.
    \begin{enumerate}
        \item $Vol(v_1,\dots, \lambda v_i, \dots, v_n) = \lambda Vol(v_1,\dots, v_n)$
        \item Забьём на то, что вектора должны лежать по разные стороны от плоскости
            и запишем свойство в общем виде: $Vol(v_1,\dots, v_i + \hat{v_i}, \dots, v_n) = 
            Vol(\dots, v_i, \dots ) + Vol(\dots, \hat{v_i} \dots)$.
        \item $Vol(\dots, v_i, \dots, v_i, \dots) = 0$
    \end{enumerate}
\end{definition}
\begin{remark}
    Заметим, что в случае $n = 1$ $Vol$ является линейным. Это намекает нам, что необходимо
    новое определение.
\end{remark}
\begin{definition}
    $V_1,\dots, V_k, U$~--- век про-ва $L: V_1\times\dots\times V_k\mapsto U$.
    Если $L$ удовлетворяет свойствам 1, 2 прошлого определения, то оно называется полилинейным.
    В случае $L: V\times\dots\times V\mapsto K$. $K$ называется полилинейной формой на $V$.
\end{definition}
\begin{definition}
    Свойство кососимметричности и симметричности.
     \begin{enumerate}
         \item $L$~--- симметричное, если $L(v_1,\dots, v_i, \dots, v_j, \dots v_n) = L(v_1,\dots, v_j,\dots, v_i,\dots, v_n)$.
         \item $L$~--- кососимметричное, если $L(\dots, v_i, \dots, v_i, \dots) = 0$
    \end{enumerate}
\end{definition}
\begin{definition}
    $n = \dim V$. $L: V\times\dots\times V \mapsto K$ полилинейное и кососимметричное.
    тогда $L$~--- форма объёма. Такое всегда существует(но вырожденное!), например нулевое
\end{definition}
\begin{motivation}
    Теперь вопрос стоит о классификации и определении свойств.
\end{motivation}
Далее для простоты будем говорить о ситуации $L: V\times\dots\times V\mapsto K$, иначе
пришлось бы выбирать в каждом $V_i$ свой базис. Структурно доказательства бы не изменились.
\begin{definition}
    Пусть $K$~--- поле, $V$~--- векторное пространство.\\ 
    Тогда $Hom_k(V,\dots, V,K)$~--- множество всех полилинейных отображений
    из $V\times\dots\times V\mapsto K$.
    Более того, оно является векторным пространством. Так как сумма полилинейных
    отображений полилинейное.
\end{definition}
\begin{remark}
    $\dim Hom_k(\underbrace{V,\dots, V}_{l}, K) = (\dim V)^l$ 
\end{remark}
\begin{proof}
    $e_1,\dots, e_n$~--- базис $V$.
    Пусть $L: V^{\times l} \mapsto K$. 
    Знаем, что$v_i = \sum\limits_{j}^{}{[v_i]_je_j}$, откуда следует равенство.
    $L(v_1\dots, v_l) = \sum\limits_{1\le j_1,\dots,j_l\le n}{L(e_{j_1}, \dots, e_{j_l})
    \prod\limits_{s=1}^{l}[v_s]_{j_s}}$.\lab{тут было какое-то произведение в конце формулы}
    Давайте распишем пример для случая, где у нас два вектора:
    \[
        L(v_1, v_2) = \left[v_1 = \sum\limits_{j_1=1}^{n}{[v_1]_{j_1}\cdot e_{j_1}};
        v_2 = \sum\limits_{j_2=1}^{n}{[v_2]_{j_2}\cdot e_{j_2}}\right]=
        \sum\limits_{j_1=1}^{n}{[v_1]_{j_1}L(e_{j_1}, v_2)} =
        \sum\limits_{j_1,j_2=1}^{n}{[v_1]_{j_1}[v_2]_{j_2}L(e_{j_1}e_{j_2})}
    \]
    \lab{дослушать, как это доказывается\обобщается}
\end{proof}
\begin{statement}\leavevmode
    \begin{enumerate}
        \item
            $L: V^{\times l}\mapsto K$~--- кососимметричное полилинейное отображение, то
            оно удовлетворяет условию $L(\dots, u,\dots, v,\dots) = -L(\dots,v,\dots,u,\dots)$.
        \item 
            $L~$~--- полилинейное, $L(\dots, u,\dots,v,\dots) = -L(\dots, v,\dots, u,\dots)$.
            $char K \not= 2 \Rightarrow L$~--- кососимметричное.
    \end{enumerate}
\end{statement}
\begin{proof}\leavevmode
    Докажем $\ola$:\\
    Посмотрим на ситуацию: $L(\dots, u + v, \dots, u + v,\dots) = 
    L(\dots, u,\dots, u,\dots) + L(\dots, u, \dots, v,\dots) + 
    L(\dots, v,\dots,u,\dots)+L(\dots, v,\dots, v) = L(\dots, u,\dots, u,\dots) + 
    L(\dots, v,\dots, v,\dots)$. В правой
    части равенства первое сокращается с третьим по свойству\lab{свойство?}

    Докажем $\ora $:\\
    $L(\dots, v,\dots,v,\dots) = -L(\dots,v,\dots,v) \Rightarrow 
    2L(\dots,v,\dots,v,\dots) = 0$.
\end{proof}
\begin{motivation}
    \lab{чего хотим?} 
\end{motivation}
\begin{statement}
    $L$~--- кососимметричная полилинейная форма, тогда:
    \[
        L(e_1,\dots, e_l) = L(\sigma(i_1),\dots, e_{\sigma(l)})\cdots sgn(\sigma),
    \]
    где $sgn(\sigma)$~--- количество транспозиций в перестановке $\sigma$.
\end{statement}
\begin{proof}
    Очевидно по свойству кососимметричности формы $L$, если знать 
    определения и свойства знака перестановки.
\end{proof}
\subsection{Знак перестановки(чётность)}
\begin{definition}
    $S_n$~--- множество перестановок на $n$ элементах.
\end{definition}
\begin{definition}
    Пусть $\sigma\in S_n$. Тогда  
    \[
        sgn\sigma = \prod\limits_{n\ge i\ge j\ge 1}\frac{\sigma(i) - \sigma(j)}{i - j}=
        \prod\limits_{i\not=j}\frac{\sigma(i) - \sigma(j)}{i - j}
    \]
    .
\end{definition}
\begin{definition}
    Инверсией назовём пару $(i,j): i > j, \sigma(i) < \sigma(j)$.
\end{definition}
\begin{remark}
    $sgn(-1)^{\text{количество инверсий}}$
\end{remark}
\begin{theorem}
    $sgn\sigma \tau = sgn \sigma\cdot sgn\tau$.
\end{theorem}
\begin{proof}
    \[
        sgn\sigma\tau = \prod\limits_{i > j} \frac{\sigma\tau(i) - \sigma\tau(j)}{i - j}=
        \prod\limits_{i > j} \frac{\sigma\tau(i) - \sigma\tau(j)}{i - j}\cdot
        \frac{\tau(i)-\tau(j)}{\tau(i)-\tau(j)}=
        \left(\prod\limits_{i > j} \frac{\sigma\tau(i)- \sigma\tau(j)}{\tau(i)-\tau(j)}\right)\cdot sgn\tau=
        sgn\sigma + sgn \tau
        .
    \]
    Первое равенство по определению. Второе очевидно. Третье равенство мы 
    просто заметили, что $\frac{\tau(i)-\tau(j)}{i-j} = sgn\tau$ по определению.
\end{proof}
\begin{statement}\leavevmode
    \begin{enumerate}
        \item $sgn\left((12)\right) = -1$
        \item $sgn\sigma = sgn\sigma^{-1}$    
        \item $sgn\left((i,j)\right) = sgn\left((12)\right) = -1$
    \end{enumerate}
\end{statement}
\begin{proof}\leavevmode
    \begin{enumerate}
        \item Очевидно
        \item По свойству гомоморфизма групп.
            $sgn\sigma\cdot sgn\sigma^{-1} = sgn(\sigma\sigma^{-1}) = sgn(id) = 1$.
        \item
        $sgn\left((i,j)\right) = sgn\left(g^{-1}(12)g\right) = sgn\left((12)\right)$, где $g\in S_n\colon g(i) = 1, g(j) = 2$. 

    \end{enumerate}
\end{proof}
\begin{follow}\leavevmode
    \begin{enumerate}
        \item
            $sgn\sigma = (-1)^{\text{кол-во транспозиций в $\sigma$}}$.
        \item
            $sgn\sigma = (-1)^{\text{кол-во циклов чётной длины в каноническом разложении $\sigma$}}$\\
            Ps. Каноническим разложением назовём разложение $\sigma$ на циклы 
            $\sigma =(123)(45)(6789)$.
        \item
            $(a_1,\dots, a_k) = (a_1,a_2)\dots(a_{k-1},a_k)$\\
            То есть любой цикл размера $k$ это композиция $k-1$ транспозиций.
    \end{enumerate}
\end{follow}
\begin{statement}
    $L$~--- кососимметричное полилинейное, $e_1,\dots,e_n$~--- базис $V$. Тогда:
    \[
        L(v_1,\dots, v_l) = \sum\limits_{1\le j_1<\dots<j_l\le n}{
    \left(\sum\limits_{\sigma\in S_l}{sgn\sigma \prod\limits_{s=1}^{l}{[v_s]_{j_\sigma(s)}}}\right)L(e_{j_1},\dots,e_{j_l})
        }
    \]
    \lab{чего}
\end{statement}
\begin{remark}
    \lab{??}
    В случае $l = n$ формула приобретает вид:
    \[
        L(v_1,\dots, v_n) = L(e_1,\dots, e_n)\cdot \sum\limits_{\sigma\in S_n}^{}{
            sgn\sigma\prod_{S = 1}{X_{S,\sigma(s)}}
        }
    \]
\end{remark}
\begin{definition}
    $A\in M_n(K)$, тогда определителем $A$ назовём: $detA = \sum\limits_{\sigma\in S_n}
    {sgn\sigma \prod\limits_{i = 1}^{n}{a_{i, \sigma(i)}}}$
\end{definition}
% \begin{theorem}[Свойства определителя] \leavevmode
%     \begin{enumerate}
%         \item
%             $det: M_n(K)\mapsto K$~--- невырожденная форма объёма.
%         \item
%             $det(AB) = detA \cdot detB$.
%         \item
%              $det \left(\begin{array}{c|c}
%                      A & B\\
%                      \hline
%                      0 & D
%              \end{array}\right) = det A \cdot det C$ 
%          \item
%              $\abs{det} = Vol$, при  $K = \R$
%     \end{enumerate}
% \end{theorem}
