\section{Лекция 28.03}
\begin{motivation}
    Хотим понять что мы хотим от объёма в общем случае и понять каким наборам
    аксиом он должен удовлетворять.
\end{motivation}

Давайте пока думать о частном случае, а именно: $\R^n$.
$\vol(E_n)$~--- объём единичного куба. Давайте выпишем свойства, которые мы в итоге хотим видеть от объема в общем случае.
\begin{enumerate}
     \item $\vol(E_n) = 1$
     \item $\vol(v_1,\dots, \lambda v_i, \dots, v_n) = \abs{\lambda} \vol(v_1,\dots, v_i,\dots, v_n)$ 
        \begin{figure}[h]
            \centering
            \incfig[0.5]{1}
            \caption{Пример данного свойства для $\mathbb{R}^2$}
        \end{figure}
     \item  Подумаем о случае $\R^3$. Вспомним каким образом в школе вычисляют объём в подобных ситуациях.
        Ответ: умножают основание на высоту и радуются жизни. Давайте попробуем это описать на языке векторов,
        что приведёт нас к ещё одному свойству, необходимому для понятия объёма.
        
        \begin{figure}[H]
            \centering
            \incfig{2}
            \caption{Пример данного свойства для $\mathbb{R}^3$}
        \end{figure}

        Пусть $\langle v_1, v_2 \rangle$ в нашем случае является основанием. Тогда
        расстояние от оставшегося вектора $v_3$ до $\langle v_1, v_2 \rangle$ будет являться высотой.
        Теперь представим $v_3$ сумму двух. Например $v_3 = u_1 + u_2$.
        Тогда, пользуясь школьной формулой объёма параллелепипеда очевидным образом выводится равенство:
        \[
            \vol(v_1, v_2, u_1) + \vol(v_1, v_2, u_2) = \vol(v_1, v_2, v_3)
        \] 
         Но тут есть подвох связанный с разложением $v_3$ на $u_1, u_2$. А именно: если $u_1, u_2$ лежат по разные стороны от плоскости
         $\langle v_1, v_2 \rangle$, то равенство не выполнено.

         В многомерье нам тоже хочется иметь это удобное свойство, поэтому запишем
         список наших прихотей следующую формулу.
         \[
             \vol(v_1,\dots, v_i + \hat{v_i}, \dots, v_n) = \vol(\dots, v_i, \dots) + \vol(\dots,\hat{v_i},\dots)
         \]
         верно, если $v_i, \hat{v_i}$ лежат по одну сторону от $\langle v_1, \dots, v_{i - 1}, v_{i + 1}, \dots, v_n\rangle$.
     \item $\vol(v_1,\dots, v_i + \lambda v_j, \dots, v_j, \dots, v_n) 
         = \vol(v_1,\dots, v_i,\dots, v_j, \dots, v_n)$ 
        \begin{figure}[H]
            \centering
            \incfig[0.5]{3}
            \caption{Пример данного свойства для $\mathbb{R}^2$}
            \label{fig:3}
        \end{figure}
     \item $\vol(v_1,\dots, v_i, \dots, v_i, \dots, v_n) = 0$.
\end{enumerate}

Наиболее интересными свойствами являются: 2,3,5, просто потому что остальные могут быть выведены из этих.
Теперь, наконец то, дадим уже корректное определение объема в общем случае.

\begin{definition}
    $\vol: V\times \dots \times V \mapsto K$, где $K$~--- поле, а $\dim V = n$.
    \begin{enumerate}
        \item $\vol(v_1,\dots, \lambda v_i, \dots, v_n) = \abs{\lambda} \vol(v_1,\dots, v_n)$
        \item $\vol(v_1,\dots, v_i + \hat{v_i}, \dots, v_n) = 
            \vol(\dots, v_i, \dots ) + \vol(\dots, \hat{v_i} \dots)$,
            если $v_i, \hat{v_{i + 1}}$ лежат по одну сторону от подпространства образованного остальными векторами.
        \item $\vol(\dots, v_i, \dots, v_i, \dots) = 0$
    \end{enumerate}
\end{definition}
\begin{remark}
    Заметим, что в случае $n = 1$ $\vol$ является линейным. Это намекает нам, что необходимо
    новое определение.
\end{remark}
\begin{definition}
    $V_1,\dots, V_k, U$~--- век про-ва $L: V_1\times\dots\times V_k\mapsto U$.
    Если $L$ удовлетворяет свойствам 1, 2 прошлого определения, то оно называется полилинейным.
    В случае $L: V\times\dots\times V\mapsto K$. $K$ называется полилинейной формой на $V$.
\end{definition}
\begin{definition}
    Свойство кососимметричности и симметричности.
     \begin{enumerate}
         \item $L$~--- симметричное, если $L(v_1,\dots, v_i, \dots, v_j, \dots v_n) = L(v_1,\dots, v_j,\dots, v_i,\dots, v_n)$.
         \item $L$~--- кососимметричное, если $L(\dots, v_i, \dots, v_i, \dots) = 0$
    \end{enumerate}
\end{definition}
\begin{definition}
    $n = \dim V$. $L: V\times\dots\times V \mapsto K$ полилинейное и кососимметричное.
    тогда $L$~--- форма объёма. Такое всегда существует(но, возможно, вырожденное!), например тождественный ноль.
\end{definition}
\begin{motivation}
    Теперь вопрос стоит о классификации и определении свойств.
\end{motivation}
Далее для простоты будем говорить о ситуации $L: V\times\dots\times V\mapsto K$, иначе
пришлось бы выбирать в каждом $V_i$ свой базис. Структурно доказательства бы не изменились.

\begin{definition}
    Пусть $K$~--- поле, $V$~--- векторное пространство.\\ 
    Тогда $Hom(V,\dots, V,K)$~--- множество всех полилинейных отображений
    из $V\times\dots\times V\mapsto K$.
    Более того, оно является векторным пространством. Так как сумма полилинейных
    отображений~--- полилинейное отображение.
\end{definition}
\begin{statement}[О размерности пространства полилинейных отображений]\leavevmode
    \begin{enumerate}
        \item \[
                L(v_1,\dots, v_n) = \sum\limits_{1\le j_1, \dots, j_l \le n}^{}{L(e_{j_1}, \dots, e_{j_l})\prod\limits_{s=1}^{l}{[v_s]_{j_s}}}
            \]
        \item \[
                \dim Hom(\underbrace{V,\dots, V}_{l}, K) = (\dim V)^l
            \]
    \end{enumerate}
\end{statement}
\begin{proof}
    Возьмём $e_1,\dots, e_n$~--- базис $V$, и посмотрим как работает произвольное
    полилинейное отображение $L: V^{l} \mapsto K$ на произвольном наборе векторов $v_1,\dots, v_n$.

    Распишем каждый $v_i$ следующим образом: $v_i = \sum\limits_{j=1}^{n}{\underbrace{[v_i]_j}_{\in K}\underbrace{e_j}_{\in V}}$.
    Давайте распишем доказательство для случая $l=2$. Просто дважды используется полилинейность $L$.
    \[
        L(v_1, v_2) = \left[v_1 = \sum\limits_{j_1=1}^{n}{[v_1]_{j_1}\cdot e_{j_1}};
        v_2 = \sum\limits_{j_2=1}^{n}{[v_2]_{j_2}\cdot e_{j_2}}\right]=
        \sum\limits_{j_1=1}^{n}{[v_1]_{j_1}L(e_{j_1}, v_2)} =
        \sum\limits_{j_1,j_2=1}^{n}{[v_1]_{j_1}[v_2]_{j_2}L(e_{j_1}e_{j_2})}
    \]

    Тогда, обобщая, получим следующее равенство:
    \begin{equation}
        L(v_1,\dots, v_n) = \sum\limits_{1\le j_1, \dots, j_l \le n}^{}{L(e_{j_1}, \dots, e_{j_l})\prod\limits_{s=1}^{l}{[v_s]_{j_s}}} \label{eq:7:1}
    \end{equation} 
    Получается, что любое полилинейное отображение однозначно задаётся своими значениями на базисных векторах $e_1,\dots, e_n$.
    Значит $\dim Hom(V^l, K) = (\dim V)l$, так как можно выбрать любой из $\dim V$ базисных векторов пространства $V$ на
    $l$ позициях.
\end{proof}
\begin{statement}[Свойство кососимметричного отображения]\leavevmode
    \begin{enumerate}
        \item
            $L: V^l\mapsto K$~--- кососимметричное полилинейное отображение, то
            оно удовлетворяет условию $L(\dots, u,\dots, v,\dots) = -L(\dots,v,\dots,u,\dots)$.
        \item 
            $L~$~--- полилинейное, $L(\dots, u,\dots,v,\dots) = -L(\dots, v,\dots, u,\dots)$.
            $char K \not= 2 \Rightarrow L$~--- кососимметричное.
    \end{enumerate}
\end{statement}
\begin{proof}\leavevmode\\
    \begin{enumerate}
        \item
            По определению кососсиметричности очевидна следующая цепочка равенств:
            \[
                \begin{gathered}
                    0 = L(\dots, u + v, \dots, u + v,\dots) =\\=
                    \underbrace{L(\dots, u,\dots, u,\dots)}_{=0} + L(\dots, u, \dots, v,\dots) + 
                    L(\dots, v,\dots,u,\dots)+\underbrace{L(\dots, v,\dots, v, \dots)}_{=0}=\\=
                    L(\dots, u,\dots, v,\dots) + L(\dots, v,\dots, u,\dots)
                \end{gathered}
            \]
        \item
            Для любого $v\in V$ выполнено:
            \[
                \begin{gathered}
                    L(\dots, v,\dots,v,\dots) = -L(\dots,v,\dots,v,\dots)
                    \\\Updownarrow\\
                    2L(\dots,v,\dots,v,\dots) = 0 
                    \\\underset{char K \not= 2}{\Updownarrow}\\
                    L(\dots, v,\dots,v,\dots) = 0
                \end{gathered}
            \]
    \end{enumerate}
\end{proof}
\begin{motivation}
    Хотим вывести аналог \hyperref[eq:7:1]{формулы для симметричного отображения}, но в отношении кососимметричного отображения.
\end{motivation}
\begin{statement}
    $L$~--- форма, тогда:
    \[
        L(v_1,\dots, v_l) = L(v_{\sigma(1)},\dots, v_{\sigma(l)})\cdot sgn(\sigma),
    \]
    где $\sigma$~--- произвольная перестановка, а $sgn(\sigma)$~--- количество транспозиций в перестановке $\sigma$.
\end{statement}
\begin{proof}
    Очевидно по свойству кососимметричности формы $L$, если знать 
    определения и свойства знака перестановки(а именно как знак связан с количеством транспозиций),
    о которых пойдёт речь в следующем разделе.
\end{proof}
\subsection{Знак перестановки(чётность)}
\begin{definition}
    $S_n$~--- множество перестановок на $n$ элементах.
\end{definition}
\begin{definition}
    Пусть $\sigma\in S_n$. Тогда  
    \[
        sgn\sigma = \prod\limits_{1 \le i < j \le n}\frac{\sigma(i) - \sigma(j)}{i - j}
    \]
\end{definition}
\begin{definition}
    Инверсией назовём пару $(i,j)\colon i > j, \sigma(i) < \sigma(j)$.
\end{definition}
\begin{statement}
    \[
        sgn (\sigma) = (-1)^{\text{количество инверсий в $\sigma$}}
    \]
\end{statement}
\begin{proof}
    Докажем просто пользуясь определением $sgn(\sigma)$ и заметив пару фактов.

    Ключевое замечание: модуль числителя и модуль знаменателя произведения 
    $\prod\limits_{1 \le i < j \le n}\frac{\sigma(i)-\sigma(j)}{i - j}$ очевидно
    равны так как каждая пара индексов в числителе и в знаменателе встречается по одному разу.
    
    Второй момент для понимания этого факта заключается в том, что в числителе
    будет столько отрицательных множителей, сколько инверсий содержится в перестановке $\sigma$, ну а в
    числителе просто все множители положительны.
\end{proof}
\begin{theorem}[Мультипликативность знака перестановки]
    $sgn\sigma \tau = sgn \sigma\cdot sgn\tau$.
\end{theorem}
\begin{proof}
    \[
        sgn\sigma\tau = \prod\limits_{i > j} \frac{\sigma\tau(i) - \sigma\tau(j)}{i - j}=
        \prod\limits_{i > j} \frac{\sigma\tau(i) - \sigma\tau(j)}{i - j}\cdot
        \frac{\tau(i)-\tau(j)}{\tau(i)-\tau(j)}=
        \left(\prod\limits_{i > j} \frac{\sigma\tau(i)- \sigma\tau(j)}{\tau(i)-\tau(j)}\right)\cdot sgn\tau=
        sgn\sigma \cdot sgn \tau
        .
    \]
    Первое равенство по определению. Второе очевидно. Третье равенство мы 
    просто заметили, что $\frac{\tau(i)-\tau(j)}{i-j} = sgn\tau$ по определению.
\end{proof}
\begin{statement}[Основные свойства перестановки]\leavevmode
    \begin{enumerate}
        \item $sgn\left((12)\right) = -1$
        \item $sgn\sigma = sgn\sigma^{-1}$    
        \item $sgn\left((i,j)\right) = sgn\left((12)\right) = -1$
    \end{enumerate}
\end{statement}
\begin{proof}\leavevmode
    \begin{enumerate}
        \item Очевидно, так как у нас всего одна транспозиция.
        \item 
            Понимаем, что $sgn(id) = 1$, после чего воспользуемся только что доказанной 
            \hyperref[thm:Мультипликативность знака перестановки]{мультипликативностью знака перестановки}.
            $sgn\sigma\cdot sgn\sigma^{-1} = sgn(\sigma\sigma^{-1}) = sgn(id) = 1$.
        \item
        $sgn\left((i,j)\right) = sgn\left(g^{-1}(12)g\right) = sgn\left((12)\right)$, где $g\in S_n\colon g(i) = 1, g(j) = 2$.
        В последнем равенстве использовался предыдущий пункт.

    \end{enumerate}
\end{proof}
\begin{follow}\leavevmode
    \begin{enumerate}
        \item
            $sgn\sigma = (-1)^{\text{кол-во транспозиций в $\sigma$}}$.
            Так как знак произведения циклов это произведение знаков и знак любой транспозиции $-1$.
        \item
            $sgn\sigma = (-1)^{\text{кол-во циклов чётной длины в каноническом разложении $\sigma$}}$\\
            Ps. Каноническим разложением назовём разложение $\sigma$ на циклы 
            $\sigma =(123)(45)(6789)$.
            Так как в цикл размера $k$ это на самом деле $k - 1$ транспозиция.
            $(a_1,\dots, a_k) = (a_1,a_2)\dots(a_{k-1},a_k)$\\
    \end{enumerate}
\end{follow}
Вернёмся к теме полилинейных отображений.

\begin{motivation}
    Хотим получить более сильный аналог 
    \hyperref[stm:О размерности пространства полилинейных отображений]{утверждения о размерности пространства полилинейных отображений}
    для формы.
\end{motivation}
\begin{statement}[О пространстве форм объёма]\leavevmode
    \begin{enumerate}
        \item
            $L$~--- форма, $e_1,\dots,e_n$~--- базис $V$. Тогда:
            \[
                L(v_1,\dots, v_l) = \sum\limits_{1\le j_1<\dots<j_l\le n}{L(e_{j_1},\dots,e_{j_l})
            \sum\limits_{\sigma\in S_l}\left(sgn\sigma \prod\limits_{s=1}^{l}{[v_s]_{j_{\sigma(s)}}}\right)
                }
            \]
        \item
            Любая форма однозначно задаётся одним числом(или элементом поля $K$).
    \end{enumerate}
\end{statement}
\begin{proof}
    Так как форма является частным случаем полилинейного отображения, то по
    \hyperref[stm:О размерности пространства полилинейных отображений]{утверждению о размерности пространства полилинейных отображений}
    верна формула:
    \[
    \begin{gathered}
        L(v_1,\dots, v_n) = \sum\limits_{1\le j_1, \dots, j_l \le n}^{}{L(e_{j_1}, \dots, e_{j_l})\prod\limits_{s=1}^{l}{[v_s]_{j_s}}}\\
        [\text{по свойству формы сократим все слагаемые с множителем } L(\dots, u, \dots, u,\dots)=0]\\
        L(v_1,\dots, v_n) = \sum\limits_{1\le j_1 < \dots < j_l \le n}^{}{L(e_{j_1}, \dots, e_{j_l})
        \sum\limits_{\sigma\in S_l}\left(sgn\sigma \prod\limits_{s=1}^{l}{[v_s]_{j_{\sigma(s)}}}\right)}\\
        [\text{так как $l=n$ по свойству формы объёма}]\\
        L(v_1,\dots, v_n) = L(e_1,\dots, e_n)\cdot \sum\limits_{\sigma\in S_n}^{}
        \left(sgn\sigma\prod_{s = 1}^{n}{[v_s]_{\sigma(s)}}\right)
    \end{gathered}
    \] 
    Получается, что любая форма объёма однозначно задаётся одним числом(в формуле это $L(e_1,\dots, e_n)$) 
    и всегда вычисляется однотипно.
\end{proof}
\begin{definition}
    $A\in M_n(K)$, тогда определителем $A$ назовём:
    \[
        \det A = 
        \sum\limits_{\sigma\in S_n} {sgn\sigma \prod\limits_{i = 1}^{n}{a_{i, \sigma(i)}}}
    \]
\end{definition}
\begin{remark}
    Определитель можно воспринимать как стандартную форму объёма
    для $K^n$. Собственно, про эту связь и будет разговор в следующих лекциях.
\end{remark}
