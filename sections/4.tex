\section{Лекция 07.03}
\subsection{Ранг линейного отображения}
\begin{definition}
    Будем называть линейной системой координат изоморфизм между каким-либо векторным пространством
    $V$ и пространством столбцов $K^n$. (Далее изоморфизм будет обозначаться как $V \simeq K^n$).
\end{definition}
\begin{remark}
    Пусть $L: U\mapsto V$~--- линейное отображение. 
    Тогда $Im L \leq V$.
\end{remark}
\begin{definition}
    Определим ядро линейного отображения: $Ker L = \{x\in U \mid Lx = 0\}$. 
\end{definition}
\begin{remark}
    Пусть $L: U\mapsto V$~--- линейное отображение. 
    Тогда $Ker L \leq U$.
\end{remark}
\begin{definition}
    Рангом линейного отображения назовём размерность образа линейного отображения.
    $rk L = \dim Im L$
\end{definition}
\begin{theorem}[О подходящем выборе базиса]
    Пусть $L: U\mapsto V$, $k = rk L$. 
    Тогда $\exists e_1,\dots,e_n$~--- базис $U$ такой, что
    $Le_1,\dots, Le_k$~--- базис $Im L$, а $e_{k+1},\dots,e_{n}$~--- базис $Ker L$
\end{theorem}
\begin{proof}
    Обозначим за $m=n - \dim Ker L$.
    Далее построим $e_{m+1},\dots, e_n$~--- базис $Ker L$.
    Тут имеется ввиду, что мы строим базис с нуля, но индексируем немного странным
    образом в связи с тем, чтобы далее было удобно показать, что на самом деле $m = k$.

    Теперь пользуясь \hyperref[thm:О дополнении до базиса]{теоремой} дополним наш базис
    при помощи любой порождающей системы $U$ до базиса $U$. Теперь для доказательства
    исходной теоремы нам необходимо доказать, что $e_1,\dots, e_m$~--- базис $Im L$.
    Будем делать это по определению, а именно докажем порождаемость и линейную независимость.

    \begin{itemize}
        \item Порождаемость.\\
            Заметим, что $Le_1,\dots,Le_n$~--- порождает $L$(просто из-за того, что $e_1,\dots,e_n$~--- базис $U$).
            Но все $Le_{m+1}, \dots, Le_n$ равны нулю, а это значит, что эта система равносильна
            $Le_1,\dots,Le_m$, откуда следует, что $Le_1,\dots, Le_m$~--- порождающая $L$.
        \item Линейная независимость.\\
            Для доказательства этой части возьмём произвольную линейную комбинацию равную 0 и докажем её 
            тривиальность(то, что $\lambda_i = 0 \forall i$).

            $\sum\limits_{1}^{m}{\lambda_i Le_i} = 0$. Пользуясь линейностью $L$ равенство можно записать
            в виде: $L\sum\limits_{1}^{m}{\lambda_ie_i}$. Заметим, что сумма просто по определению
            лежит в $Ker L$. Это, в свою очередь значит, что эта же сумма представима как линейная
            комбинация элементов из базиса $Ker L$, то есть:  
            $\sum\limits_{1}^{m}{\lambda_ie_i} = \sum\limits_{m+1}^{n}{\lambda_ie_i}$. 
            Теперь можно перенести оба выражения в левую часть, чтобы чётко видеть, что мы получили
            просто линейную комбинацию базиса $U$, равную нулю: 
            $\sum\limits_{1}^{m}{\lambda_ie_i} - \sum\limits_{m+1}^{n}{\lambda_ie_i} = 0$. 
            Но так, как $e_1,\dots, e_n$~--- базис $U$, то такое возможно только когда
            $\lambda_i = 0\forall i$, что и требовалось доказать.
    \end{itemize}
    Значит $Le_1,\dots,Le_m$~--- базис пространства $Im L$.
\end{proof}
\begin{follow}
    $L: U\mapsto V$~--- линейное отображение, причём $\dim U > \dim V$.
    Тогда выполнено:  $Ker L > 0$.\\
\end{follow}
\begin{follow}
    $L: U\mapsto V$~--- линейное отображение, тогда $rk L + \dim Ker L = \dim U$
\end{follow}
\begin{remark}
    Давайте посмотрим на достаточно интересный частный случай формулы выше, а именно, когда
    $\dim Ker L = 0$. Тогда верно следующее утверждение: $L\text{~--- инъекция} \Leftrightarrow \dim Ker L = 0$.
\end{remark}
\begin{proof}\leavevmode\\
    $\ora$:\\
        Докажем от обратного, пусть $L$~--- инъективное, но $\dim Ker L > 0$.
        Тогда верно: $\exists u\in U\colon \lambda Lu = 0 \forall \lambda$.
        Значит можно просто взять $u, 2u$ и окажется, что  $Lu = L2u = 0$, 
        противоречие с инъективностью $L$.
    \\$\ola$:\\
        Пусть $L$ не инъективное и $\dim Ker L == 0$, тогда $\exists u_1, u_2\in U\colon Lu_1 = Lu_2\Rightarrow
        L(u_1 - u_2) = 0 \Rightarrow \dim Ker L > 0$, противоречие. 
\end{proof} 
\begin{follow}
    Пусть $\dim U = \dim V = n$, $L: U\mapsto V$~--- линейное.
    Тогда верна следующая цепочка утверждений:
    $$L\text{~--- инъективное} \Leftrightarrow \dim Ker L = 0 \Leftrightarrow rk L = n \Leftrightarrow 
    \dim Im L = V \Leftrightarrow L\text{~--- сюръективное}$$.
    Получили нечто очень похожее на обобщённый принцип Дирехле для обыкновенной теории множеств.
\end{follow}

\subsection{Матрица системы уравнений}
\begin{motivation}
    Давайте лучше поймём какими наборами уравнений могут задаваться подпространства.
\end{motivation}
\begin{definition}
    Допустим $U\leq K^n$. Тогда будем говорить, что $U$ задаётся системой уравнений, если
    $\exists A \in M_{d\times n}(K)\colon Ker A = U$.
    Сразу возникает вопрос: какие должны быть ограничения на $d$, чтобы задать в точности $U$.
\end{definition}
\begin{statement}
    Пусть $m = \dim U, U\leq K^n$ Тогда $U$ задаётся не менее $n - m$ уравнениями, при этом
    точно есть способ задать $U$ при помощи ровно $n - m$ уравнений.
\end{statement}
\begin{proof}
    Обозначим $d$~--- количество уравнений в системе. Докажем первую часть утверждения от обратного. 
    Пусть $A\in M_{d\times n}(K)\colon Ker Ax = U$, причём $d < n - m$.
    Зная, что $Ax\colon K^n\mapsto K^d$ уже можно получить какую-то оценку на $d$, 
    а именно $\dim Ker Ax \geq n - d$. Тогда можно перенести $n$ в левую часть и получить:
    $d \geq n - Ker Ax =  n - m$. Получили противоречие.

    Для того, чтобы доказать существование просто построим линейное отображение $L: K^n\mapsto K^{n-m}\colon
    Ker L = U$. Для этого удобнее всего воспользоваться 
    \hyperref[thm:Линейное отображение однозначно задаётся двумя базисами]{теоремой о единственности линейного
    отображения, которое задано парой базисов}.

    Построим базис пространства $K^n$ следующим образом:  $e_1,\dots,e_m$~--- базис $U$,
    $e_{m + 1},\dots, e_n$~--- его дополнение до $K^n$, такое существует по \hyperref[thm:О дополнении до базиса]{теореме}.
    Заметим, что $Le_1 = \dots = Le_m = 0$, так как иначе не выполняется условие: $Ker L = U$. 
    Тогда, оставшиеся $e_{m + 1},\dots, e_n$~--- должны быть базисом $Im L = K^{n - m}$.
    Самый простой способ добиться этого~--- отправить их в стандартный базис.
    
    Таким образом, мы определили наше линейное отображение на всех элементах базиса
    следующим образом $Le_1 = \dots = Le_m = 0$, $Le_{m + 1} = E_1, \dots, Le_n = E_{n - m}$,
    где $E_i$~--- $i$ый элемент стандартного базиса. А значит, по 
    \hyperref[thm:Линейное отображение однозначно задаётся двумя базисами]{теореме}
    мы однозначно определили искомое линейное отображение $L$.
\end{proof}
\subsection{Про матрицы линейных отображений}
\begin{definition}
    Пусть $v\in V$,  $f_1,\dots, f_n$~--- базис $V$.
    Тогда $[v]_f =
    \begin{pmatrix}
        \lambda_1\\ \vdots\\ \lambda_n
    \end{pmatrix}$, где $\lambda_i$~--- коэффициенты в следующем разложении:
    $v = \sum\limits_{1}^{n}{\lambda_if_i}$
\end{definition}
\begin{definition}
    Пусть $L: U\mapsto V$~--- линейное отображение, $e_1,\dots, e_n$~--- базис $U$, 
    $f_1,\dots, f_m$~--- базис $V$. $A = \Big([Le_1]_f, \dots, [Le_n]_f\Big)\in M_{m\times n}(K)$
    $A$~--- матрица линейного отображения $L$. Будем обозначать её следующим образом $[L]_f^e$
\end{definition}
\begin{remark}
    Хочется отметить, что выше мы по факту поняли, что существует биекция между
    матрицами нужного размера и всеми линейными отображениями при условии, что мы
    знаем какие-либо базисы наших пространств.
\end{remark}
\begin{motivation}
    Будем развивать эту историю с матрицей отображения и поймём зачем она вообще нужна.
\end{motivation}
\begin{statement}
    $L: U\mapsto V, e_1,\dots, e_n\text{~--- базис }U, f_1,\dots, f_m\text{~--- базис }V$.
    $A \in M_{m\times n}(K)\colon \forall u\in U\colon A[u]_e = [Lu]_f$ равносильно тому,
    что $A$~--- матрица линейного отображения $L$.
\end{statement}
\begin{proof}\leavevmode\\
    $\ola$:\\
    Раскроем по определению: $[L]^e_f[u]_e=[Lu]_f$.
    Можно просто технически расписать по определению, но давайте проведём доказательство,
    которое даст лучшее понимание происходящего. Ключевой момент: $[u]_e, [Lu]_f$ в этой
    формуле лучше воспринимать просто как две биекции в $K^n$ и в $K^m$ соответственно.

    Тогда мы тут имеем систему следующих отображений:
    $K^n \simeq U \xrightarrow{L} V \simeq K^m$,
    $K^n\xrightarrow{[L]^e_f} K^m$.

    Зная, что, благодаря 
    \hyperref[thm:Линейное отображение однозначно задаётся двумя базисами]{теореме}
    , можно задать линейное отображение, только на базисных векторах, посмотрим на то, чем
    будет являться $E_1\in K^n$ при переходе из $K^n$ в $K^m$ разными методами.
    \[
        E_1\in K^n \xrightarrow{([x]_e)^{-1}} e_1 \in U \xrightarrow{L} Le_1 \in V
        \xrightarrow{[x]_f} [Le_1]_f\in K^m
    .\]
    С другой стороны:
    \[
        E_1\in K^n\xrightarrow{[L]^e_fx} \Big([L]^e_f\Big)_{*, 1} = [Le_1]_f\in K^m
    .\] 
    Очевидно, что подобный трюк можно провернуть с любым $E_i$, а значит мы для всех
    элементов базиса $K^n$ верно $[L]^e_f[u]_e=[Lu]_f$, откуда следует, что это равенство
    верно и на всём пространстве $K^n$.\\
    $\ora$:
    Для доказательства в эту сторону теперь достаточно показать, что такая матрица $A$~---
    единственная. Пусть $\exists A,B\in M_{m\times n}(K)\colon A \not = B$, но при этом 
    $Ax = Bx \forall x\in U\simeq K^n$, сразу видно, что тут противоречие с $A\not=B$.
\end{proof}
\subsection{Связь умножения матриц и композиции линейных отображений}
